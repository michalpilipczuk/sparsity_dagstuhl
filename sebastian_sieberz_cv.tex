\documentclass[polish,10pt]{article}
\usepackage{amsfonts}
\usepackage{graphicx}
\usepackage{amsmath}
\usepackage{amsthm}
\usepackage{amstext}
\usepackage{amsopn}
\usepackage{latexsym}
\usepackage[utf8]{inputenc}
%\usepackage{kpfonts}
\usepackage[T1]{fontenc}
%\usepackage[polish,czech]{babel}
\usepackage[section]{algorithm}
\usepackage{latexsym}
\usepackage{color}
\usepackage{comment}
\usepackage{url}
%\input epsf
\pagestyle{empty}

\newcommand{\ignore}[1]{}
\newcommand{\CC}{{\bf{C:}}\quad}
\newcommand{\JJ}{{\bf{J :}}\quad}

\newcommand{\heading}[1]{
  \vfill
  \vspace{0.2cm}
  \hrule height .2mm  \vspace{1mm}
  \noindent {\large \textbf{#1}}\\ \vspace{-3mm} \hrule height .2mm
  \vspace{0.3cm}
}

\newcommand{\smallheading}[1]{
  \vfill
  \vspace{0.2cm}
  \hrule height .2mm  \vspace{1mm}
  \noindent {\small \textbf{#1}}\\ \vspace{-3mm} \hrule height .2mm
  \vspace{0.3cm}
}

\addtolength{\hoffset}{-28mm}
\addtolength{\textwidth}{56mm}
\addtolength{\voffset}{-20mm}
\addtolength{\textheight}{28mm}

\hyphenpenalty=100000

%\pagenumbering{none}\newcounter{pcount}
\newcounter{pcount}

\newcommand{\paperplain}[2]{
\noindent\begin{tabular}{@{}p{0.7cm} @{\hspace{2mm}} p{16.8cm}}
[\arabic{pcount}\refstepcounter{pcount}\label{#1}] & #2
\end{tabular}
\vskip 0.2cm
}

\newcommand{\paper}[4]{
\noindent\begin{tabular}{@{}p{0.7cm} @{\hspace{2mm}} p{16.8cm}}
[\arabic{pcount}\refstepcounter{pcount}\label{#1}] & #2, \\
& {\em{#3}},\\
& #4
\end{tabular}
\vskip 0.2cm
}

\newcommand{\paperlncs}[5]{
\noindent\begin{tabular}{@{}p{0.7cm} @{\hspace{2mm}} p{16.8cm}}
[\arabic{pcount}\refstepcounter{pcount}\label{#1}] & #2, \\
& {\em{#3}},\\
& #4 \\
& #5
\end{tabular}
\vskip 0.2cm
}


\newcommand{\paperk}[5]{
\noindent\begin{tabular}{@{}p{0.7cm} @{\hspace{2mm}} p{16.8cm}}
[\arabic{pcount}\refstepcounter{pcount}\label{#1}] & #2, \\
& {\em{#3}},\\
& #4 \\[0.1cm]
& {\bf{Conf.}}: #5
\end{tabular}
\vskip 0.2cm
}

\newcommand{\paperklncs}[6]{
\noindent\begin{tabular}{@{}p{0.7cm} @{\hspace{2mm}} p{16.8cm}}
[\arabic{pcount}\refstepcounter{pcount}\label{#1}] & #2, \\
& {\em{#3}},\\
& #4 \\[0.1cm]
& {\bf{Conf.}}: #5\\
& \hskip 0.96cm #6
\end{tabular}
\vskip 0.2cm
}

\newcommand{\papercm}[5]{
\noindent\begin{tabular}{@{}p{0.7cm} @{\hspace{2mm}} p{16.8cm}}
[\arabic{pcount}\refstepcounter{pcount}\label{#1}] & #2, \\
& {\em{#3}},\\
& #4 \\[0.1cm]
& {\bf{Note}}: #5
\end{tabular}
\vskip 0.2cm
}

\newcommand{\papercmlncs}[6]{
\noindent\begin{tabular}{@{}p{0.7cm} @{\hspace{2mm}} p{16.8cm}}
[\arabic{pcount}\refstepcounter{pcount}\label{#1}] & #2, \\
& {\em{#3}},\\
& #4 \\
& #5 \\[0.1cm]
& {\bf{Note}}: #6
\end{tabular}
\vskip 0.2cm
}

\newcommand{\papercmk}[6]{
\noindent\begin{tabular}{@{}p{0.7cm} @{\hspace{2mm}} p{16.8cm}}
[\arabic{pcount}\refstepcounter{pcount}\label{#1}] & #2, \\
& {\em{#3}},\\
& #4 \\[0.1cm]
& {\bf{Conf.}}: #6\\
& {\bf{Note}}: #5
\end{tabular}
\vskip 0.2cm
}

\newcommand{\papercmklncs}[7]{
\noindent\begin{tabular}{@{}p{0.7cm} @{\hspace{2mm}} p{16.8cm}}
[\arabic{pcount}\refstepcounter{pcount}\label{#1}] & #2, \\
& {\em{#3}},\\
& #4 \\[0.1cm]
& {\bf{Conf.}}: #6\\
& \hskip 0.96cm #7\\
& {\bf{Note}}: #5
\end{tabular}
\vskip 0.2cm
}


\begin{document}

\heading{Personal Information}

\begin{small}
\noindent
\begin{tabular}{@{\hspace{0cm}}l @{\hspace{10mm}} p{13cm}}
{\bf Name and surname} & Sebastian Siebertz\\[0.1cm]
{\bf Date and place of birth} & February 20$^{\textrm{th}}$, $1984$, Bergisch-Gladbach, Germany\\[0.1cm]
{\bf Citizenship} & German\\[0.1cm]
{\bf Address} & Institut f\"ur Informatik, Chair for Algorithm Engeneering, Humbold-Universit\"at zu Berlin, \\
& Johann-von-Neumann-Haus, Rudower Chausse 25, 
D-12489 Berlin-Adlershof, Germany\\[0.1cm]
{\bf Phone number} & +49 30 3093 3010\\[0.1cm]
{\bf E-mail} & \verb+siebertz@informatik.hu-berlin.de+\\[0.1cm]
{\bf Website} & \verb+http://www.mimuw.edu.pl/~siebertz+\\[0.1cm]
{\bf Research interests} & 
Logic in computer science, structural graph theory, 
parameterized complexity theory
\end{tabular}
\end{small}

\heading{Education and academic career}
\begin{small}
\noindent
\begin{tabular}{@{}p{3cm} @{\hspace{2mm}} p{14.7cm}}
since October 2018 & Postdoc at the Chair for Algorithm Engeneering 
 at Humboldt-Universit\"at zu Berlin \\[0.2cm]
2016 --- 2018 & Marie Sk\l odowska-Curie Fellow (supported by the National Science Centre of Poland and the European Union's Horizon 2020 research and 
innovation  programme) at the Institute of Informatics of the University of
    Warsaw. \\[0.2cm]
2015 --- 2016 & Postdoc at the 
chair for Logic and Semantic at Technical University Berlin (TU Berlin), Germany.\\[0.2cm]
2011 --- 2015 & Doctoral research fellow (PhD position) at the 
chair for Logic and Semantic at
Technical University Berlin (TU Berlin), Germany, working under the supervision of Prof.\ Stephan Kreutzer.
PhD thesis titled \emph{Nowhere Dense Classes of Graphs: Characterisations and Algorithmic Meta-Theorems} defended in September 2015.  Parental leave from February 2014 --- February 2015.\\[0.2cm]
2004 --- 2011 & Studies in computer science at RWTH 
Aachen University. Diploma in computer science defended
in April 2011. 

\end{tabular}
\end{small}

\heading{Leadership and participation in grants}
\begin{small}
\noindent
\begin{tabular}{@{}p{3cm} @{\hspace{2mm}} p{14.7cm}}
2016 --- 2018 & Principal investigator in POLONEZ grant ``Algorithmic Structure Theory for Sparse Graphs'', funded by the National Science Center of Poland from the Horizon 2020 programme funds, Marie Sk\l{}odowska-Curie actions.\\[0.2cm]
\end{tabular}
\end{small}


%\heading{Selected scientific awards}
%\begin{small}
%\noindent
%\begin{tabular}{@{}p{1.8cm} @{\hspace{2mm}} p{15.7cm}}
%2016 & ERCIM Cor Baayen Award 2016. Prize awarded annually to one promising young researcher in computer science and applied mathematics.\\[0.1cm]
%2015, 2016 & START stipend granted by the Foundation for Polish Science (FNP), awarded with a~distinguishment for the highest ranked applications.\\[0.1cm]
%2015 & Witold Lipski Prize for the best young researchers working in computer science in Poland.\\[0.1cm]
%2015 & Stipend of the Ministry of Science and Higher Education of the Republic of Poland for outstanding young~researchers.\\[0.1cm]
%2014 & Meltzer Prize for Young Researchers (Meltzerprisen for yngre forskere), awarded for the achievements in 2013.
%\end{tabular}
%\end{small}
%
%\pagebreak

\heading{Selected talks}
\begin{small}
\noindent
\begin{tabular}{@{}p{1.5cm} @{\hspace{2mm}} p{16cm}}
Jul 2018 & Talk \emph{On the number of types in sparse graphs} at
the ACM/IEEE Symposium on Logic in Computer Science (LICS 2018),
Oxford, UK.  \\[0.1cm]
Jan 2018 & Talk \emph{Polynomial Kernels and Wideness Properties of
Nowhere Dense Graph Classes} at the ACM-SIAM Symposium on Discrete Algorithms (SODA 2017), Barcelona, Spain. \\[0.1cm]
Sep 2017 & Invited talk \emph{First-Order Model-Checking} at the 
parameterized complexity summer school, September 1--3, 2017, Vienna, Austria.\\[0.1cm]
Jul 2014 & Invited talk \emph{Deciding first-order properties of
 nowhere dense graphs} at the Midsummer Combinatorial Workshop XX, July 28 -- Aug 1, 2014, Charles University, Prague, Czech Republic.
\end{tabular}
\end{small}

\pagebreak
\heading{Organization of scientific events}
\begin{small}
\noindent
\begin{tabular}{@{}p{1.6cm} @{\hspace{2mm}} p{15.9cm}}
Jul 2017 & Co-organizer (with Michał Pilipczuk) of a satellite workshop of ICALP 2018 on algorithms and structure for sparse graphs. Warsaw, Poland.\\[0.1cm]
Sep 2015 & Local co-organizer of \emph{Computer Science Logic} (CSL 2015), Berlin, Germany.
\end{tabular}
\end{small}

\heading{Community service}
\begin{small}
\noindent {\bf{Co-supervised Bachelor students}}: Moritz Zielke, 
Thesis titled \emph{Empirical Evaluation of Splitter-Game Based Algorithms for the Dominating Set Problem.}
Alexander Court, Thesis titled \emph{Empirical Evaluation of Approximation Algorithms for Graph Colouring Numbers.}
Frank Dehne, Thesis titled \emph{Empirical Evaluation of Approximation Algorithms for Directed Tree-Width.}
\vskip 0.1cm
\noindent {\bf{Program Committee member of}}: CSL 2018, Highlights of Logics Games and Automata 2018.
\end{small}


\heading{Selected publications connected with the proposal's topic}

\setcounter{pcount}{1}

\begin{footnotesize}

\paper{mc}{Martin Grohe, Stephan Kreutzer, Sebastian Siebertz}
{Deciding first-order properties of nowhere dense graphs}{Journal of the ACM, 2017}

\paperlncs{wide-stab}{Jakub Gajarsk\'y, Stephan Kreutzer, Jaroslav Ne\v{s}et\v{r}il, Patrice Ossona de Mendez, Micha\l{} Pilipczuk, Sebastian Siebertz, Szymon Toru\'nczyk}
{First-order interpretations of bounded expansion classes}{Proceedings of the 45$^{\text{th}}$ International Colloquium on Automata, Languages, and Programming, ICALP 2018}
{Volume 107 of LIPIcs, Schlo\ss{} Dagstuhl --- Leibniz-Zentrum f\"ur Informatik, 2018}

\paper{wide-stab}{Michał Pilipczuk, Sebastian Siebertz, Szymon Toruńczyk}
{On the number of types in sparse graphs}{Proceedings of the 23$^{\textrm{rd}}$ Annual ACM/IEEE Symposium on Logic in Computer Science, LICS 2018}


\paperlncs{nei-comp}{Kord Eickmeyer, Archontia Giannopoulou, Stephan Kreutzer, O-joung Kwon, Michał Pilipczuk, Roman Rabinovich, Sebastian Siebertz}
{Neighborhood complexity and kernelization for nowhere dense classes of graphs}{Proceedings of the 44$^{\text{th}}$ International Colloquium on Automata, Languages, and Programming, ICALP 2017}
{Volume 80 of LIPIcs, Schlo\ss{} Dagstuhl --- Leibniz-Zentrum f\"ur Informatik, 2017}

\paperlncs{bounded-exp}{Pål Grønås Drange, Markus S. Dregi, Fedor V. Fomin, Stephan Kreutzer, Daniel Lokshtanov, Marcin Pilipczuk, Michał Pilipczuk, Felix Reidl, Fernando Sánchez Villaamil, Saket Saurabh, Sebastian Siebertz, Somnath Sikdar}
{Kernelization and Sparseness: the case of Dominating Set}{Proceedings of the 33$^{\textrm{rd}}$ International Symposium on Theoretical Aspects of Computer Science, STACS 2016}
{Volume 47 of LIPIcs, Schlo\ss{} Dagstuhl --- Leibniz-Zentrum f\"ur Informatik, 2016}
\end{footnotesize}

\vspace*{5cm}
\end{document}
