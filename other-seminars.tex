\paragraph*{Other seminars, workshops, and projects.}
There have been several events addressing different aspects of sparsity in combinatorics and algorithm design.
This demonstrates a high interest of various communities in mathematics and computer science
to foster further multidisciplnary interaction on topics covered by this proposal and
we believe that the environment of Schloss Dagstuhl is ideal to achieve this goal.
\begin{itemize}
\item A one-day satellite workshop on algorithms and structure of sparse graphs was organized in connection with the 44th International Colloquium on Automata, Languages, and Programming, ICALP 2017, held in Warsaw in July 2017.
\item A workshop on structural sparsity, logic, and algorithms was organized by Anuj Dawar, Zden\v ek Dvo\v{r}\'ak, and Daniel Kr\'a\v{l} at the University of Warwick in June 2018. This was a follow-up event of a workshop on similar topics organized at the same place in December 2016 by Daniel Kr\'a\v{l} and Ranko Lazi\'c. See \url{https://warwick.ac.uk/fac/sci/maths/people/staff/daniel_kral/strlogalg/}
and \url{https://warwick.ac.uk/fac/sci/maths/people/staff/daniel_kral/alglogstr}.
\item A one-month research project ``Research in Paris'' on {\em Structural Sparisty} was hosted by Institut Henri Poincar\'e (Paris) during May 2018, which gathered J. Ne\v set\v ril, P. Ossona de Mendez, F. Reidl, S.~Siebertz, and B. Sullivan 
\url{http://www.ihp.fr/en/activities/rip/forthcoming}.
\item In the autumn of 2018, Charles University in Prague is organizing a six-week program (DocCourse) on structural sparsity in connection with model theory offering tutorial lectures to senior undergraduate and graduate students. See \url{https://iuuk.mff.cuni.cz/events/doccourse2018/}.
\end{itemize}
We are not aware of any previous Dagstuhl seminars directly overlapping with the subject of our proposal.
