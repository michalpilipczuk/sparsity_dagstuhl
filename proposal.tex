\documentclass[10pt]{article}

\usepackage{fullpage}
\usepackage[all=normal,bibliography=tight]{savetrees}

\usepackage{amsmath,amsthm,amssymb}
\usepackage{hyperref} 
\usepackage{todonotes}
\usepackage{xspace}
\usepackage{ifthen}
\usepackage[czech, english]{babel}
\selectlanguage{english} % small trick to get $\v{l} for Dan ....

\newcommand{\ourtitle}{Sparsity}
\newcommand{\email}[1]{\href{mailto:#1}{\nolinkurl{#1}}}
\newcommand{\organizer}[7]{%
\noindent\begin{tabular}{p{3cm}l}
\textbf{Name:} & #1 \\
\textbf{Affiliation:} & #2 \\
\textbf{Address:} & #3 \\
\textbf{Phone:} & #4 \\
\textbf{Fax:} & #5 \\
\textbf{Email:} & \email{#6} \\
\textbf{Homepage:} & \url{#7}
\end{tabular}\par\vspace{4mm}}

\newcommand{\invitee}[6]{%
\noindent\begin{tabular}{p{3cm}l}
\textbf{Name:} & #1 \\
\textbf{Affiliation:} & #2 \\
\textbf{Email:} & \email{#3} \\
\ifthenelse{\equal{#4}{}}{}{\textbf{Homepage:} & \url{#4} \\}
\textbf{Topics:} & #5
\ifthenelse{\equal{#6}{}}{}{\\ \textbf{Tags:} & #6}
\end{tabular}\par\vspace{2mm}}

\renewcommand\thesection{\Alph{section}}

\title{{\large{Proposal for a Dagstuhl seminar on}}\\
  {\huge{\textbf{\ourtitle}}}}

\author{
  Daniel Kr\'a\v{l} \and
  Micha\l{} Pilipczuk \and
  Sebastian Siebertz\and
  Blair D. Sullivan 
}

\date{\today}

\begin{document}
\maketitle

\begin{abstract}
The concept of \emph{sparsity} in combinatorics
aims at capturing abstract notions of uniform sparsity of graphs,
as well as more general relational structures.
The main goal is to obtain understanding why, and to what extent,
structural properties related to the sparsity of a given structure
can be used to develop sparsity-based methods and algorithms.
Since the pioneering work of Ne\v{s}et\v{r}il and Ossona de Mendez from the late 2000's,
a huge body of research has shown that the two main sparsity concepts --- \emph{bounded expansion} and \emph{nowhere-denseness} ---
have deep connections to classic notions from algorithm design, combinatorics and model theory, and
they can be used to obtain powerful structural and algorithmic results.
It is the synergy of algorithm design, combinatorics and model theory that
makes studying sparsity mathematically exciting and brings so many computer science applications.
One of the many evidences of the fundamental nature of the concepts of bounded expansion and nowhere-denseness is that
these two concepts constitute the boundary of the computational tractability for many natural classes of problems, 
most notably the first order model-checking.

The aim of the proposed seminar is to bring together researchers working on various aspects of sparsity in their own fields,
in order to facilitate the exchange of ideas, methods and questions between different communities.
An important part of the seminar will be the discussion of the (still fledgling) area of real-life applications of
sparsity-based methods, where theory and practice could meet.
%The theory of {\em{sparsity}} studies abstract notions of uniform sparsity for classes of graphs, as well as more general logical structures.
%The main goal of the theory is to understand why, and to what extent, sparsity of a given structure can be used to describe its properties, 
%and to develop tools helpful for designing sparsity-based methods.
%Since the work of Ne\v{s}et\v{r}il and Ossona de Mendez, who laid solid foundations of the theory in the late 2000's, 
%a huge body of work has shown that the two main notions --- {\em{bounded expansion}} and {\em{nowhere denseness}} ---
%have deep connections with classic concepts from combinatorics, model theory, and algorithm design, and can be used to obtain new, powerful results in these areas.
%It is the synergy of these three fields that makes sparsity a mathematically rich and exciting theory, which is currently under rapid development.
%The notions of bounded expansion and nowhere denseness also constitute borders of computational tractability for natural classes of problems, 
%most notably for the model-checking problem for first-order logic; this witnesses the fundamental nature of the studied concepts.
%
%The aim of the proposed seminar is to bring together researchers working on various aspects of sparsity, in order to facilitate 
%the exchange of ideas, methods, and questions between different communities. An important part of the seminar will be the discussion
%of the (still fledgling) area of real-life applications of sparsity-based methods, where theory and practice could meet.
\end{abstract}

\section{Metadata}

\begin{enumerate}
\item {\bf{Organizers}}:
\begin{itemize}
\item Daniel Kr\'a\v{l}, Masaryk University in Brno, Czech Republic, \email{dkral@fi.muni.cz};
\item Micha\l{} Pilipczuk, University of Warsaw, Poland, \email{michal.pilipczuk@mimuw.edu.pl};
\item Sebastian Siebertz, Humboldt University in Berlin, Germany, \email{siebertz@mimuw.edu.pl};
\item Blair D. Sullivan, North Carolina State University, Raleigh, USA, \email{blair_sullivan@ncsu.edu}.
\end{itemize}
\item {\bf{Title}}: \ourtitle
\item {\bf{Type}}: Dagstuhl Seminar
\item {\bf{Size and duration}}: Large (45 participants) and Long (5 days)
\item {\bf{Classification}}: TODO
\item {\bf{Keywords}}: TODO
\end{enumerate}

\section{Proposal text}

Blah blah blah~\cite{sparsity}

\paragraph*{Introduction.}
It was realized already in the early days of computer science that structures (networks, databases, etc.) that are {\em{sparse}} appear ubiquitously in applications.
The sparsity of input can be used in a variety of ways, e.g. to design efficient algorithms. This motivates a theoretical study of the abilities and limitations of sparsity-based methods.
However, a priori it is not clear how to even define sparsity formally.
Focusing on graphs with bounded maximum degree seems too restrictive, as real-life networks tend to have high-degree hubs while retaining a moderate overall number of edges,
while only requiring a constant upper bound on the average degree is too relaxed, as it allows the existence of dense substructures.
Another way is to concentrate on classes with topological constraints, e.g. planar graphs or, more generally, proper minor-closed classes,
or to generalize the example of bounded-degree networks to classes with bounded {\em{degeneracy}} (equivalently, bounded {\em{arboricity}}).
While these ideas inspired multiple results, it can be argued that they did not lead to the development of a robust mathematical theory.

In the late 2000s, Ne\v{s}et\v{r}il and Ossona de Mendez~\cite{NesetrilM08,NesetrilM08a,NesetrilM08b} proposed a different approach:
to introduce new definitions of uniform, structural sparsity for classes of graphs that would generalize the currently considered settings,
and to develop a toolbox of sparsity-based methods for such sparse classes.
The central notions they introduced are classes of {\em{bounded expansion}} and {\em{nowhere dense}} classes.
In a nutshell, a class of graphs $\Cc$ has bounded expansion if by contracting constant-radius connected subgraphs into single vertices
in any graph from $\Cc$ one cannot obtain graphs with arbitrary high average degree; for nowhere denseness, we only insist that in this way one cannot obtain arbitrary large cliques.
Thus, bounded expansion and nowhere denseness can be thought of uniform sparsity that persists under local modifications, modelled by constant-radius contractions.
Every class of bounded expansion is nowhere dense, and every class that excludes a fixed topological minor --- in particular every proper minor-closed class and every class of bounded maximum degree ---
has bounded expansion. Thus, these concepts are indeed broader than the previously considered settings, while sparse networks appearing in applications indeed tend to come from
classes of bounded expansion~\cite{DemaineRRVSS14}.

It quickly turned out that the proposed notions can be used to build a mathematical theory of sparse graph classes that offers a wealth of tools, leading to new techniques and powerful results.
Indeed, the last~10 years have seen a great progress in the area, witnessed by many publications in top venues.
It is particularly remarkable that the concepts of bounded expansion and nowhere denseness can be connected
to fundamental ideas from multiple other fields of computer science, often in a surprising way, and thus there are several complementary viewpoints on the subject.
On one hand, foundations of the area are grounded in {\em{structural graph theory}}, which aims at describing structure in graphs through various decompositions and auxiliary parameters.
On the other hand, nowhere denseness seems to delimit the border of expressibility and algorithmic tractability of first-order logic, which provides a link to {\em{finite model theory}}.
Finally, there is a fruitful transfer of ideas to and from the field {\em{algorithm design}}: sparsity-based methods can be used to design new, efficient algorithms,
and classic techniques of designing algorithms on sparse inputs inspire new combinatorial results on sparse graphs.
It is the synergy of these three fields that makes the theory of sparse graphs so exciting.
In the next paragraphs we explain how the recent progress can be viewed through the eyes of researchers working in these areas.

The aim of the proposed seminar is to ``stir in the pot'' and invite researchers working with structurally sparse structures to share their experience, questions, and ideas.
An important secondary goal is to facilitate discussion on practical applications of (so far) theoretical methods that were recently developed.
%We believe that the recent rapid progress in the area calls for capitalizing on the momentum.

\paragraph*{Eyes of a combinatorialist.}

%Graph classes with restricted structure (such as minor closed classes) and classes based on various notions of width
%play an important role in algorithm design.  Classical results from the theory of graph minors have recently been
%extended to more general classes of graphs, e.g. those avoiding immersions and subdivisions, which provided new insights
%on their structure including the existence of small-size separators.  The introduction of graph classes with bounded
%expansion and nowhere-dense classes has led to a new robust way of understanding sparsity with many implications
%in combinatorics and computer science.  An important direction seeks to transfer methods from the study of sparse
%classes to non-sparse classes, such as graphs of bounded clique-width and their generalisations.

\paragraph*{Eyes of a model theorist.}
Model theory  is the study of classes of mathematical structures from the perspective of mathematical logic, and this study mainly concerns infinite structures.
A classification of theories based on those whose models can be classified and those whose models are too complicated to classify has been initiated by Shelah, who  drew two important dividing lines: NIP vs dependent and stable vs unstable \cite{shelah1990classification}.

Finite model theory, which focuses on finite structures,  diverges significantly from the study of infinite structures in both the problems studied and the techniques used, and mainly grew out of computer science applications.  This led to the development of strong tools to study logics over finite structures, which helped to answer many questions about  complexity theory, databases, formal languages, algorithms, and many more. This also led to the notion of {\em tame} classes of finite structures, which behave well.  In this line Dawar \cite{Dawar2010} introduced the notion of a \emph{uniform quasi-wide class} and proved 
     that the homomorphism preservation theorem holds when relativized to a uniformly quasi-wide class of finite structures. It was later shown 
by Ne\v{s}et\v{r}il and Ossona de Mendez  that the uniform quasi-wide classes are exactly nowhere dense classes \cite{ND_logic}.

The  structural and algorithmic properties of  nowhere dense classes  are deeply connected to  the model theory notions of independence and stability, as witnessed in particular by the following two fundamental results. The first asserts that for a monotone class of graphs, the notions of nowhere dense class, NIP class, and stable class are equivalent \cite{adler2014interpreting}. The second states that nowhere dense classes are the most general monotone classes for which first-order model checking is fixed-parameter tractable \cite{grohe2017deciding}. This connection has already proved its efficiency in transferring techniques and notions from model theory to combinatorics (see for instance \cite{pilipczuk2018number}) and, conversely, from combinatorics to finite model theory (see for instance \cite{rossman2008homomorphism}).

Numerous algorithmic applications of structural properties of the graphs in a nowhere dense class (or, more restrictively in a bounded expansion class) have appeared, including   a near-optimal kernelization algorithm for the distance-$r$ dominating set problem for the graphs in a nowhere dense class \cite{eickmeyer2016neighborhood}.

Guided by the model theory intuition, it is natural to expect that some of the techniques and results proved for monotone sparse classes may extend to hereditary structurally sparse classes, that is possibly dense classes of low structural complexity. In this direction, it has been proved that model checking is fixed parameter tractable on map graphs \cite{eickmeyer2017fo} and on interpretations of classes of graphs with bounded maximum degree \cite{gajarsky2016new} and that transductions of bounded expansion classes of graphs admit low shrub-depth decompositions, a dense analog of low tree-depth decompositions \cite{gajarsky2018first}. 


\paragraph*{Algorithms.}
\paragraph*{Applied eyes of an algorithm designer.}

Over the last twenty years, the field of network science
has burgeoned, developing new methods for complex network data arising
in diverse fields including social networks, bioinformatics, quantum computing,
transportation, and healthcare. Surprisingly, few tools from structural graph theory have been assimilated into
their arsenal. In part, this is due to the theoretical nature of
much of the related literature on parameterized graph algorithms
and a lack of cross-pollination of the research
communities.

There has been some work on making structure-based approaches more practical, mostly focused on
bounded treewidth and planar graphs. When working with bounded treewidth, one significant
challenge is finding low-width tree decompositions -- here, work by Bodlaender and Koster (e.g.~\cite{bodlaender2006-tw, koster2001-tw})
set the stage for the PACE challenge (https://pacechallenge.org). More recently, there have also been
experimental evaluations of treewidth-based algorithms for solving downstream optimization problems like
Vertex Cover and Hamiltonian Cycle~\cite{ziobro2018hamcycle-tw, alber2005vc-tw}. In an even more
applied setting, recently treewidth solvers were shown to be competitive for finding contraction orderings
for tensor networks used in quantum computing simulations~\cite{dumitrescu2018tensors}. There has also been some
work on parallel algorithms for bounded treewidth graphs, including a distributed memory implementation of a
Maximum Weighted Independent Set solver~\cite{sullivan2013paralleltd}.
Planar graphs have also attracted more practical interest, in part due to their natural representation of problems with geographical
constraints (e.g.~\cite{alber2001-planar,schmidt2009-planarvision}).

Since the introduction of structural sparsity and bounded expansion,
there has been a resurgence in interest, in part because of work showing that
this class includes many real-world networks (e.g.~\cite{DemaineRRVSS14}).
Similar to algorithms for bounded treewidth, pipelines for structurally sparse graph classes
require computation of a ``decomposition,'' such as a low-treedepth coloring. An end-to-end implementation
of the algorithm for subgraph isomorphism counting~\cite{obrien2017concuss} revealed that the
number of colors used in the current constant-factor approximation algorithms was causing a significant
practical bottleneck. Subsequently, there has been work on comparing algorithms and heuristics
for computing these parameters~\cite{wojciech2018quasiwide}, as well as more theoretical research
on alternative colorings that trade off treedepth for smaller coloring numbers~\cite{kun2018lincolor}.
These approaches are also starting to gain traction in interdisciplinary collaborations; a very recent
preprint shows that algorithm engineering allows the constant-factor approximation algorithm
for Distance-$r$ Dominating Set given by Dvo\v{r}\'ak to be applied to large assembly graphs in metagenomics~\cite{brown2018metagenome}.

It remains a challenge to transform efficient algorithms into practical ones (by reducing hidden constants and other trickery),
engineer scalable implementations, and forge collaborations with domain experts to ensure usability and relevance.

\paragraph*{Expected outcomes.}
The main goal of the proposed seminar is to create a platform for the exchange of tools, ideas, and questions between researchers working on different aspects of the theory.
So far, such synergy led to major developments due to the multi-disciplinary character of the field, and we hope that the seminar will foster new cross-field collaborations and lead to new results.
Besides this, the studied concepts of sparsity are very young and the new techniques are still not widely known in the related fields. 
This particularly applies to the field of algorithm design, where the toolbox seems to be applicable within multiple paradigms, whose respective communities are not aware of the new developments.
It would be desired if the seminar contributed to the visibility of the theory of sparse graph classes within neighboring areas and inspired new results based on its techniques.
Finally, the recent attempts at deploying theoretical sparsity-based methods in practice show promise and we believe that the community should strongly support applications-driven research.
We hope that the seminar will encourage more practical works such as implementation, evaluation, and fine-tuning of theoretical methods, and the usage of sparsity-based subroutines in larger systems.

\paragraph*{Structure of the seminar.}
We would like to keep the seminar oriented on research and collaboration, and therefore leave the participants plenty of time for working together.
A block of talks will be scheduled each morning, while afternoons will be kept free for collaboration.
The scientific program will be a mix of invited tutorials on important techniques, keynote lectures on recent developments, and shorter contributed talks on topics suggested by the organizers.
On the afternoon of the first day we plan an open problems session; we also would like to contact several key participants beforehand and ask them to prepare some concrete open problems.
If needed, a second open problem session will be held later during the week.

\paragraph*{Other seminars, workshops, and projects.}
There have been several events addressing different aspects of sparsity in combinatorics and algorithm design.
This demonstrates a high interest of various communities in mathematics and computer science
to foster further multidisciplnary interaction on topics covered by this proposal and
we believe that the environment of Schloss Dagstuhl is ideal to achieve this goal.
\begin{itemize}
\item A one-day satellite workshop on algorithms and structure of sparse graphs was organized in connection with the 44th International Colloquium on Automata, Languages, and Programming, ICALP 2017, held in Warsaw in July 2017.
\item A workshop on structural sparsity, logic, and algorithms was organized by Anuj Dawar, Zden\v ek Dvo\v{r}\'ak, and Daniel Kr\'a\v{l} at the University of Warwick in June 2018. This was a follow-up event of a workshop on similar topics organized at the same place in December 2016 by Daniel Kr\'a\v{l} and Ranko Lazi\'c. See \url{https://warwick.ac.uk/fac/sci/maths/people/staff/daniel_kral/strlogalg/}
and \url{https://warwick.ac.uk/fac/sci/maths/people/staff/daniel_kral/alglogstr}.
\item A one-month research project ``Research in Paris'' on {\em Structural Sparisty} was hosted by Institut Henri Poincar\'e (Paris) during May 2018, which gathered J. Ne\v set\v ril, P. Ossona de Mendez, F. Reidl, S.~Siebertz, and B. Sullivan 
\url{http://www.ihp.fr/en/activities/rip/forthcoming}.
\item In the autumn of 2018, Charles University in Prague is organizing a six-week program (DocCourse) on structural sparsity in connection with model theory offering tutorial lectures to senior undergraduate and graduate students. See \url{https://iuuk.mff.cuni.cz/events/doccourse2018/}.
\end{itemize}
We are not aware of any previous Dagstuhl seminars directly overlapping with the subject of our proposal.


\section{Invitee list}

\invitee{Isolde Adler}{University of Leeds}{UK}{I.M.Adler@leeds.ac.uk}{https://engineering.leeds.ac.uk/staff/810/Dr_Isolde_Adler}{Algorithms, logic}{junior}

\invitee{Marthe Bonamy}{LaBRI Bordeaux and CNRS}{France}{marthe.bonamy@u-bordeaux.fr}{http://www.labri.fr/perso/mbonamy/}{Combinatorics, algorithms}{female junior}

\invitee{Nicolas Bousquet}{University Grenoble Alpes and CNRS}{France}{nicolas.bousquet@grenoble-inp.fr}{https://pagesperso.g-scop.grenoble-inp.fr/~bousquen/}{Combinatorics, algorithms}{female}

\invitee{Anuj Dawar}{University of Cambridge}{UK}{anuj.dawar@cl.cam.ac.uk}{https://www.cl.cam.ac.uk/~ad260/}{Combinatorics, algorithms, logic}{}

\invitee{Reinhard Diestel}{University of Hamburg}{Germany}{ReinhardDiestel@math.uni-hamburg.de}{https://www.math.uni-hamburg.de/home/diestel/}{Combinatorics, algorithms}{}

\invitee{Zdeněk Dvořák}{Charles University, Prague}{Czechia}{ook@ucw.cz}{https://iuuk.mff.cuni.cz/~rakdver/}{Combinatorics, algorithms, logic}{}

\invitee{Eduard Eiben}{University of Bergen}{Norway}{Eduard.Eiben@uib.no}{https://www.uib.no/en/persons/Eduard.Eiben}{Algorithms}{female}

\invitee{Kord Eickmeyer}{Technical University, Darmstadt}{Germany}{eickmeyer@mathematik.tu-darmstadt.de}{https://www2.mathematik.tu-darmstadt.de/~eickmeyer/}{Algorithms, logic}{}

\invitee{Fedor Fomin}{University of Bergen}{Norway}{fomin@ii.uib.no }{http://www.ii.uib.no/~fomin/}{Algorithms}{}

\invitee{Jakub Gajarský}{Technical University, Berlin}{Germany}{jakub.gajarsky@tu-berlin.de}{http://logic.las.tu-berlin.de/Members/Gajarsky/}{Algorithms, logic}{female}

\invitee{Robert Ganian}{Technical University, Vienna}{Austria}{rganian@ac.tuwien.ac.at}{https://www.ac.tuwien.ac.at/people/rganian/}{Algorithms, logic}{female}

\invitee{Archontia Giannopoulou}{Technical University, Berlin}{Germany}{Archontia.Giannopoulou@gmail.com}{http://users.uoa.gr/~arcgian/index.html}{Combinatorics, algorithms}{female junior}

\invitee{Martin Grohe}{RWTH Aachen}{Germany}{grohe@informatik.rwth-aachen.de }{}{Combinatorics, algorithms, logic}{}

\invitee{Sariel Har-Peled}{University of Illinois in Urbana-Champaign}{USA}{sariel@uiuc.edu}{https://sarielhp.org/}{Algorithms}{}

\invitee{Petr Hliněný}{Masaryk University, Brno}{Czechia}{hlineny@fi.muni.cz}{https://www.fi.muni.cz/~hlineny/}{Algorithms, logic}{}

\invitee{Gwenaël Joret}{Free University of Brussels}{Belgium}{gjoret@ulb.ac.be}{http://di.ulb.ac.be/algo/gjoret/}{Combinatorics}{}

\invitee{Ken-ichi Kawarabayashi}{National Institute of Informatics, Tokyo}{Japan}{kkeniti@nii.ac.jp}{}{Combinatorics, algorithms}{}

\invitee{Tereza Klimošová}{Charles University, Prague}{Czechia}{tereza@kam.mff.cuni.cz}{https://iuuk.mff.cuni.cz/~tereza/}{Combinatorics}{female junior}

\invitee{Daniel Král'}{Masaryk University, Brno}{Czechia}{dkral@fi.muni.cz}{http://www.ucw.cz/~kral/}{Combinatorics, algorithms, logic}{}

\invitee{Stephan Kreutzer}{Technical University, Berlin}{Germany}{stephan.kreutzer@tu-berlin.de}{http://logic.las.tu-berlin.de/Members/Kreutzer/}{Algorithms, logic}{}

\invitee{O-joung Kwon}{Incheon National University}{South Korea}{ojoungkwon@inu.ac.kr}{http://ojkwon.com/}{Algorithms}{female}

\invitee{Aurelie Lagoutte}{University Clermont-Auvergne}{France}{aurelie.lagoutte@uca.fr}{https://fc.isima.fr/~alagoutte/}{Combinatorics}{female junior}

\invitee{Daniel Lokshtanov}{University of California in Santa Barbara}{USA}{daniello@ucsb.edu}{https://www.cs.ucsb.edu/people/faculty/lokshtanov}{Algorithms}{}

\invitee{Dániel Marx}{Hungarian Academy of Sciences (MTA SZTAKI)}{Hungary}{dmarx@cs.bme.hu }{http://www.cs.bme.hu/~dmarx/}{Combinatorics, algorithms}{}

\invitee{Claire Mathieu}{CNRS}{France}{cmathieu@di.ens.fr}{https://www.di.ens.fr/ClaireMathieu.html}{Algorithms}{}

\invitee{Piotr Micek}{Jagiellonian University, Cracow}{Poland}{piotr.micek@tcs.uj.edu.pl}{http://www.tcs.uj.edu.pl/micek}{Combinatorics}{}

\invitee{Amer Mouawad}{University of Bergen}{Norway}{amer.mouawad@gmail.com }{https://folk.uib.no/amo110/}{Algorithms}{female}

\invitee{Irene Muzi}{University of Warsaw}{Poland}{irene.muzi@gmail.com}{https://sites.google.com/site/irenemuzigraph/home}{Combinatorics, algorithms}{female junior}

\invitee{Wojciech Nadara}{University of Warsaw}{Poland}{wojtek.nadara@gmail.com }{}{Algorithms, applications}{female}

\invitee{Jaroslav Nešetřil}{Charles University, Prague}{Czechia}{nesetril@iuuk.mff.cuni.cz}{https://iuuk.mff.cuni.cz/~nesetril/en/}{Combinatorics, algorithms, logic}{}

\invitee{Jan Obdržálek}{Masaryk University, Brno}{Czechia}{obdrzalek@fi.muni.cz}{https://www.fi.muni.cz/~xobdrzal/}{Algorithms, logic}{}

\invitee{Patrice Ossona de Mendez}{CAMS Paris, CNRS, and Charles University, Prague}{France/Czechia}{patrice.ossona-de-mendez@ehess.fr }{http://cams.ehess.fr/patrice-ossona-de-mendez/}{Combinatorics, algorithms, logic}{}

\invitee{Sang-il Oum}{KAIST, Daejeon}{South Korea}{sangil@kaist.edu}{https://mathsci.kaist.ac.kr/~sangil/}{Combinatorics, algorithms}{}

\invitee{Fahad Panolan}{University of Bergen}{Norway}{Fahad.Panolan@uib.no}{https://folk.uib.no/fpa082/}{Algorithms}{female}

\invitee{Marcin Pilipczuk}{University of Warsaw}{Poland}{marcin.pilipczuk@mimuw.edu.pl}{https://www.mimuw.edu.pl/~malcin/}{Algorithms, applications}{}

\invitee{Michał Pilipczuk}{University of Warsaw}{Poland}{michal.pilipczuk@mimuw.edu.pl}{https://www.mimuw.edu.pl/~mp248287/}{Algorithms, logic}{}

\invitee{Kent Quanrud}{University of Illinois in Urbana-Champaign}{USA}{quanrud2@illinois.edu}{http://quanrud2.web.engr.illinois.edu/}{Algorithms}{female}

\invitee{Daniel Quiroz}{University of Chile, Santiago de Chile}{Chile}{dquiroz@cmm.uchile.cl}{http://www.cmm.uchile.cl/~dquiroz/}{Combinatorics}{female}

\invitee{Roman Rabinovich}{Technical University, Berlin}{Germany}{roman.rabinovich@tu-berlin.de}{http://logic.las.tu-berlin.de/Members/Rabinovich/}{Algorithms, applications}{female}

\invitee{Jean-Florent Raymond}{Technical University, Berlin}{Germany}{raymond@tu-berlin.de}{http://www.user.tu-berlin.de/jraymond/}{Combinatorics, algorithms}{female}

\invitee{Felix Reidl}{Birbeck University of London}{UK}{f.reidl@dcs.bbk.ac.uk}{https://tcs.rwth-aachen.de/~reidl/}{Combinatorics, algorithms}{female}

\invitee{Saket Saurabh}{Institute of Mathematical Sciences, Chennai}{India}{saket@imsc.res.in}{http://www.ii.uib.no/~saket/}{Algorithms}{}

\invitee{Nicole Schweikardt}{Humboldt University, Berlin}{Germany}{schweikn@informatik.hu-berlin.de}{https://www2.informatik.hu-berlin.de/~schweikn/}{Algorithms, logic}{junior}

\invitee{Luc Segoufin}{INRIA and ENS Ulm}{France}{luc.segoufin@inria.fr}{https://who.rocq.inria.fr/Luc.Segoufin/}{Algorithms, logic}{}

\invitee{Sebastian Siebertz}{Humboldt University, Berlin}{Germany}{siebertz@informatik.hu-berlin.de}{https://www.informatik.hu-berlin.de/de/institut/mitarbeiter/1691486}{Algorithms, applications}{female}

\invitee{Konstantinos Stavropoulos}{University of Hamburg}{Germany}{konstantinos.stavropoulos@uni-hamburg.de}{https://www.math.uni-hamburg.de/home/stavropoulos/}{Combinatorics}{female}

\invitee{Blair Sullivan}{North Carolina State University, Raleigh}{USA}{blair_sullivan@ncsu.edu}{https://people.engr.ncsu.edu/vbsulliv/}{Algorithms, applications}{junior}

\invitee{Dimitrios Thilikos}{LIRMM Montpellier and CNRS}{France}{sedthilk@thilikos.info}{https://www.lirmm.fr/~thilikosto/}{Combinatorics, algorithms}{}

\invitee{Szymon Toruńczyk}{University of Warsaw}{Poland}{szymtor@mimuw.edu.pl}{https://www.mimuw.edu.pl/~szymtor/}{Algorithms, logic}{}

\invitee{Torsten Ueckerdt}{Karlsruhe Institute of Technology}{Germany}{torsten.ueckerdt@kit.edu}{https://i11www.iti.kit.edu/en/members/torsten_ueckerdt/index}{Combinatorics}{female}

\invitee{Jan van den Heuvel}{London School of Economics and Political Science}{UK}{j.van-den-heuvel@lse.ac.uk}{http://www.maths.lse.ac.uk/personal/jan/}{Combinatorics}{}

\invitee{Alexandre Vigny}{University of Warsaw}{Poland}{alexandre.vigny@imj-prg.fr}{https://webusers.imj-prg.fr/~alexandre.vigny/}{Algorithms, logic}{female}

\invitee{Kristina Vušković}{University of Leeds}{UK}{K.Vuskovic@leeds.ac.uk}{https://engineering.leeds.ac.uk/staff/249/kristina_vuskovic}{Combinatorics}{junior}

\invitee{Bartosz Walczak}{Jagiellonian University, Cracow}{Poland}{walczak@tcs.uj.edu.pl}{http://www.tcs.uj.edu.pl/walczak}{Combinatorics}{}

\invitee{Daniel Weissauer}{University of Hamburg}{Germany}{TODO@TODO}{}{Combinatorics}{female}

\invitee{Paul Wollan}{University of Rome “La Sapienza”}{Italy}{wollan@di.uniroma1.it}{http://wwwusers.di.uniroma1.it/~wollan/}{Combinatorics, algorithms}{}

\invitee{David Wood}{Monash University, Melbourne}{Australia}{david.wood@monash.edu}{http://users.monash.edu.au/~davidwo/}{Combinatorics, algorithms}{}

\invitee{Marcin Wrochna}{University of Oxford}{UK}{m.wrochna@mimuw.edu.pl }{https://www.mimuw.edu.pl/~mw290715/}{Combinatorics, algorithms}{female}

\invitee{Meirav Zehavi}{Ben-Gurion University}{Israel}{zehavimeirav@gmail.com}{https://sites.google.com/site/zehavimeirav/}{Algorithms}{female junior}

\invitee{Xuding Zhu}{National Sun Yat-sen University, Kaohsiung}{Taiwan}{zhu@math.nsysu.edu.tw}{http://www.math.nsysu.edu.tw/~zhu/}{Combinatorics}{}



\section{Organizers' CVs}

\section{Proposed dates}

TODO

\bibliographystyle{abbrv}
\bibliography{references}



\end{document}
