\documentclass[10pt]{article}

\usepackage{fullpage}
\usepackage[all=normal,bibliography=tight]{savetrees}

\usepackage{amsmath,amsthm,amssymb}
\usepackage{hyperref} 
\usepackage{todonotes}
\usepackage{xspace}
\usepackage{ifthen}

\newcommand{\ourtitle}{Sparsity}
\newcommand{\email}[1]{\href{mailto:#1}{\nolinkurl{#1}}}
\newcommand{\organizer}[7]{%
\noindent\begin{tabular}{p{3cm}l}
\textbf{Name:} & #1 \\
\textbf{Affiliation:} & #2 \\
\textbf{Address:} & #3 \\
\textbf{Phone:} & #4 \\
\textbf{Fax:} & #5 \\
\textbf{Email:} & \email{#6} \\
\textbf{Homepage:} & \url{#7}
\end{tabular}\par\vspace{4mm}}

\newcommand{\invitee}[6]{%
\noindent\begin{tabular}{p{3cm}l}
\textbf{Name:} & #1 \\
\textbf{Affiliation:} & #2 \\
\textbf{Email:} & \email{#3} \\
\ifthenelse{\equal{#4}{}}{}{\textbf{Homepage:} & \url{#4} \\}
\textbf{Topics:} & #5
\ifthenelse{\equal{#6}{}}{}{\\ \textbf{Tags:} & #6}
\end{tabular}\par\vspace{2mm}}

\renewcommand\thesection{\Alph{section}}

\title{{\large{Proposal for a Dagstuhl seminar on}}\\
  {\huge{\textbf{\ourtitle}}}}

\author{
  Daniel Kr\'al' \and
  Micha\l{} Pilipczuk \and
  Sebastian Siebertz\and
  Blair D. Sullivan 
}

\date{\today}

\begin{document}
\maketitle

\begin{abstract}
The theory of {\em{sparsity}} studies abstract notions of uniform sparsity for classes of graphs, as well as more general logical structures.
The main goal is to understand why, and to what extent, sparsity of a given structure can be used to describe its properties, 
and to develop tools helpful for designing sparsity-based methods.
Since the work of Ne\v{s}et\v{r}il and Ossona de Mendez, who laid solid foundations of the theory in the late 2000s, 
a huge body of work has shown that the two main notions --- {\em{bounded expansion}} and {\em{nowhere denseness}} ---
have deep connections with classic concepts from combinatorics, model theory, and algorithm design, and can be used to obtain new, powerful results in these areas.
It is the synergy of these three fields that makes sparsity a mathematically rich and exciting theory, which is currently under rapid development.
The notions of bounded expansion and nowhere denseness also constitute borders of computational tractability for natural classes of problems, 
most notably for the model-checking problem for first-order logic; this witnesses the fundamental nature of the studied concepts.

The aim of the proposed seminar is to bring together researchers working on various aspects of sparsity, in order to facilitate 
the exchange of ideas, methods, and questions between different communities. An important part of the seminar will be the discussion
of the (still fledgling) area of real-life applications of sparsity-based methods, where theory and practice could meet.
\end{abstract}

\section{Metadata}

\begin{enumerate}
\item {\bf{Organizers}}:
\begin{itemize}
\item Daniel Kr\'al', Masaryk University in Brno, Czech Republic, \email{kral@ucw.cz};
\item Micha\l{} Pilipczuk, University of Warsaw, Poland, \email{michal.pilipczuk@mimuw.edu.pl};
\item Sebastian Siebertz, Humboldt University in Berlin, Germany, \email{siebertz@mimuw.edu.pl};
\item Blair D. Sullivan, North Carolina State University, Raleigh, USA, \email{blair_sullivan@ncsu.edu}.
\end{itemize}
\item {\bf{Title}}: \ourtitle
\item {\bf{Type}}: Dagstuhl Seminar
\item {\bf{Size and duration}}: Large (45 participants) and Long (5 days)
\item {\bf{Classification}}: TODO
\item {\bf{Keywords}}: TODO
\end{enumerate}

\section{Proposal text}

Blah blah blah~\cite{sparsity}

\paragraph*{Introduction.}

We may like to say that sparsity is viewed differently by researchers with different backgrounds and
we can then present the views. This will unify the text. It seemed to be as rather four independent parts and
we have not been sending a unifying message in my view. \emph{Written by DK.}

My biggest issue with the text at the moment is that it's not forward-looking, and doesn't emphasize what would be valuable
about bringing people from these communities together. That is, what's wrong with just letting this continue to be a
productive line of research in each of these communities? Why would a Dagstuhl be beneficial? The individual views
seem to hide the potential for advancement through interaction or collaboration.  \emph{Written by BDS.}

\paragraph*{Eyes of a combinatorialist.}
Graphs embeddable in the plane have always been understood as a prototype of a class of sparse graphs,
both from the structural and algorithmic points of view.
However, the sparsity is not linked to purely quantitative parameters such as the number of edges:
an edge-subdivision of every graph has the number of edges linear in the number of vertices
but still possess complex properties of the original graph.
A great amount of research has been concerning the investigation of questions on structural properties
implying key combinatorial and algorithmic results linked to the sparsity of plane graphs.
Many such questions has been answered by the graph minor project of Robertson and Seymour,
which is published in a sequence of 23 papers totaling over 750 pages.
They provided a structural description of graphs avoiding specific graphs as a minor;
a particular example of such class of graphs is the class of graphs embeddable in the plane.
Many concepts that are now better understood as a part of this project are of high importance in computer science:
for example, the tree-width parameter, which measures how close the structure of a graph is to that of a tree,
plays an essential role in computer science in relation to designing algorithms for sparse inputs.

The concepts of classes of bounded expansion and nowhere dense classes properly generalize classes characterized by excluding a fixed minor.
Therefore, a theory built around these ideas yields more general results and puts the previous work in a perspective.
It turns out that the notions of bounded expansion and nowhere denseness have multiple different, equivalent characterizations.
For instance, the characterization via {\em{generalized coloring numbers}} originates in the concept of bounded degeneracy and provides an algorithmically useful decomposition of the graph with small ``local''
separators~\cite{Zhu09}
{\em{Low tree-depth decompositions}} break the graph into a constant number of well-behaved pieces that can be treated somewhat independently~\cite{NesetrilM08a}.
{\em{Uniform quasi-wideness}} formalizes the intuition that in a sparse graphs there should be many vertices that are far from each other~\cite{nevsetvril2010first}.
{\em{Neighborhood complexity}} aims at measuring the sparsity in terms of the complexity of set systems defined by constant-radius balls in a graph~\cite{ReidlVS19}.
The {\em{Splitter game}} is a game characterization of nowhere denseness that provides a shallow hierarchical decomposition of a graph into local clusters~\cite{grohe2017deciding}.

Each of the above concepts is a tool: it offers a different combinatorial handle that can be used to view the problem from a different viewpoint.
The combination of all these ideas provides a powerful toolbox of sparsity-based methods, using which one can often provide conceptually simpler, yet more general reasonings than 
with the help of techniques originating in the graph minor project. 
An ingredient that is inherently lacking is the existence of a global tree-like decompositions,
which exist in the cases of graphs with excluded minor~\cite{GM16}, topological minor~\cite{GroheM15}, or immersion~\cite{Wollan15}.

So far, the combinatorial research within the theory of sparse graph classes focused on identifying and understanding different characterizations of sparsity, as well as connections between them.
In particular, quantitative relations between different parameters involved are an important topic of study.
As we discuss in the next sections, much of this research was driven by algorithmic and model-theoretical applications.


\begin{comment}
Recent years have brought substantial extensions of the results stemming from the graph minor project and
a new fascinating view on sparse graphs through lenses of the concepts of graph classes with bounded expansion and
nowhere dense graph classes introduced by Ne\v set\v ril and Ossona de Mendez.
Many natural classes of sparse graphs, in particular minor-closed classes and classes of graphs with bounded degrees,
have bounded expansion and so fall into the framework developed by Ne\v est\v ril and Ossona de Mendez.
This framework turned out to be very robust since the graph classes captured by these notions
can be characterized in equivalent ways using several different means: low-treedepth colorings,
uniform quasi-wideness, existence of small separations, generalized coloring numbers, etc.
It soon became obvious that the concepts developed in this framework play an important role
in relation to algorithm design, in particular to tractability of first-order model checking~\cite{KreutzerD09,DvorakKT13,GroheK11}.
This line of research culminated with a general first order model checking result of Grohe et al.~\cite{grohe2017deciding},
which in turn provided a new alternative description of nowhere-dense graph classes.

While the purely quantitative sparsity measures (such as the edge density) are not linked to any global structural properties,
the sparsity properties captured by the graph classes discussed above.
For example, one of the main results of the graph minor project is that
graphs avoiding a minor admit tree-like decompositions into pieces almost embeddable in surfaces of bounded genus.
Grohe and Marx~\cite{GroheM15} gave a similar description for graph classes closed under topological containment and
Dvo\v r\'ak~\cite{Dvorak12} extended these results to graph classes closed under immersions.
Motivated by problems from algorithm design,
there is an intensive research on improving the dependence on parameters involved in these structural results.
In particular,
the recent simplified proof of the structure theorem in the graph minor setting
led to surprisingly good bounds~\cite{KawarabayashiTW18},
which is of fundamental importance in relation to efficient fixed parameter tractable algorithms based on graph minors.
\end{comment}
\paragraph*{Eyes of a model theorist.}
The theory of nowhere dense graphs is intimately linked to model
theory and finite model theory. The connection to finite model
theory was first established by Dawar, who introduced the notion
of \emph{uniform quasi-wideness} and proved that the homomorphism
preservation theorem holds for finite structures that are uniformly
quasi-wide~\cite{dawar2010homomorphism}. It was later
observed by Ne\v{s}et\v{r}il and Ossona de Mendez that for graphs
the notions of uniform quasi-wideness and nowhere denseness
coincide~\cite{nevsetvril2010first}. 

In the sequel, the characterization of nowhere dense classes via
uniform quasi-wideness turned out to be very useful, especially 
for the task of algorithmically testing first-order properties of
graphs. The problem of determining whether a formula $\varphi$
of some logic $\mathcal{L}$ is true on a given structure is known 
as the \emph{model-checking problem} for $\mathcal{L}$. By proving
tractability of the model-checking problem for a logic $\mathcal{L}$
one establishes tractability for a large number of problems, namely
for all problems that can be formulated in $\mathcal{L}$. For this
reason,  such  tractability results are often called 
\emph{algorithmic meta 
theorems}. It was shown by Dvo\v{r}\'ak et al.~\cite{DvorakKT13}
that every first-order property of graphs can be decided in linear time
on every fixed class of graphs of bounded expansion, and by Grohe et 
al.~\cite{grohe2017deciding} that every first-order property of 
graphs can be decided in almost linear time on every fixed nowhere
dense class of graphs. Phrased in terms of parameterized
complexity theory, these results state that the first-order 
model-checking problem is fixed-parameter tractable, parameterized
by the length of the input formula, on every bounded bounded
expansion or nowhere dense class of graphs. On the other hand, 
nowhere dense classes that are closed under taking subgraphs 
form the border of tractability for the first-order 
model-checking problem. 
It was shown by Dvo\v{r}\'ak et al.~\cite{DvorakKT13} and
Kreutzer~\cite{kre11} that first-order model-checking on classes that
are closed under taking subgraphs and that are not nowhere
dense is as hard as on general graphs, in particular, it is 
assumed to be not fixed-parameter tractable. 
Hence, on the one hand, the above algorithmic meta theorems 
help to understand the essence of sparsity
based algorithmic techniques by abstracting from problem-specific
details. On the other hand, the corresponding hardness results show 
the limitations of these techniques beyond sparse graphs. 

Recent research aims to extend the model-checking result
beyond nowhere dense graph classes that are not closed 
under taking subgraphs. One approach is to study classes
of graphs that are \emph{structurally sparse} in the sense
that they are first-order interpretations of sparse graphs. 
In this direction, it has been proved that model checking is 
fixed parameter tractable on map graphs~\cite{eickmeyer2017fo}, 
and on interpretations of classes of graphs with bounded 
maximum degree~\cite{gajarsky2016new}. On a structural 
level, it was shown that interpretations of bounded expansion 
classes of graphs admit low shrub-depth decompositions, 
a dense analog of low tree-depth 
decompositions~\cite{gajarsky2018first}.

A second approach to extending the model-checking results 
is based on another deep connection of
nowhere denseness with classical model theory, more precisely
with \emph{stability theory}. It was observed by Adler and 
Adler~\cite{adler2014interpreting} that nowhere denseness
corresponds to the model-theoretic notion of 
\emph{superflatness}. Furthermore, it was observed that nowhere
dense classes are \emph{stable}, a key notion in Shelah's 
classification theory~\cite{shelah1990classification}. 
Surprisingly, on subgraph closed classes of graphs the notions
of stability and nowhere denseness coincide. This connection
has already proved its efficiency in transferring techniques and 
notions from model theory to combinatorics and algorithms, 
see for instance \cite{siebertz2016polynomial, 
malliaris2014regularity, pilipczuk2018number}. 

%Model theory  is the study of classes of mathematical structures from the perspective of mathematical logic, and this study mainly concerns infinite structures.
%A classification of theories based on those whose models can be classified and those whose models are too complicated to classify has been initiated by Shelah, who  drew two important dividing lines: NIP vs dependent and stable vs unstable \cite{shelah1990classification}.
%
%Finite model theory, which focuses on finite structures,  diverges significantly from the study of infinite structures in both the problems studied and the techniques used, and mainly grew out of computer science applications.  This led to the development of strong tools to study logics over finite structures, which helped to answer many questions about  complexity theory, databases, formal languages, algorithms, and many more. This also led to the notion of {\em tame} classes of finite structures, which behave well.  In this line Dawar \cite{Dawar2010} introduced the notion of a \emph{uniform quasi-wide class} and proved 
%     that the homomorphism preservation theorem holds when relativized to a uniformly quasi-wide class of finite structures. It was later shown 
%by Ne\v{s}et\v{r}il and Ossona de Mendez  that the uniform quasi-wide classes are exactly nowhere dense classes \cite{ND_logic}.
%
%The  structural and algorithmic properties of  nowhere dense classes  are deeply connected to  the model theory notions of independence and stability, as witnessed in particular by the following two fundamental results. The first asserts that for a monotone class of graphs, the notions of nowhere dense class, NIP class, and stable class are equivalent \cite{adler2014interpreting}. The second states that nowhere dense classes are the most general monotone classes for which first-order model checking is fixed-parameter tractable \cite{grohe2017deciding}. This connection has already proved its efficiency in transferring techniques and notions from model theory to combinatorics (see for instance \cite{pilipczuk2018number}) and, conversely, from combinatorics to finite model theory (see for instance \cite{rossman2008homomorphism}).
%
%Numerous algorithmic applications of structural properties of the graphs in a nowhere dense class (or, more restrictively in a bounded expansion class) have appeared, including   a near-optimal kernelization algorithm for the distance-$r$ dominating set problem for the graphs in a nowhere dense class \cite{eickmeyer2016neighborhood}.
%


\paragraph*{Theoretical eyes of an algorithm designer.} 
Combinatorial methods developed in the study of classes of sparse graphs turned out to be very useful in the design of algorithms, in particular within the framework of {\em{parameterized complexity}}.
In this area, the goal is develop algorithms whose running time is measured not only in terms of the total input size, but also secondary measures called {\em{parameters}} that govern the actual 
difficulty of the instance. The usual goal here is to show that a problem is {\em{fixed-parameter tractable}}, that is, 
design an algorithm that runs in time $f(k)\cdot n^c$, where $n$ is the input size, $k$ is the parameter (or a vector of parameters), 
$f$ is a computable function, and $c$ is a universal constant, independent of $k$.
This philosophy very well matches the toolbox of sparsity, which provides a wealth of parameters measuring structural sparsity of graphs, such as generalized coloring numbers or
low-treedepth colorings. Historically, the study of parameterized algorithms on sparse structures focused on the cases of graphs of bounded degree and on proper minor-closed classes, with a particular focuse on
planar graphs and graphs of bounded tree-width.

Note that all the abovementioned settings can be subsumed by the concepts of bounded expansion and nowhere denseness.
It turns out that for certain general type of problems, these structural sparsity notions form a very suitable context of study, yielding conceptually simpler, yet more general reasonings than 
in the case of proper minor-closed classes. Not surprisingly, the techniques are best-suited to problems with a local character:
one looks for a solution (say, subset of vertices) that admits a property that can be checked by inspecting a constant-depth neighborhood of every vertex.
Example problems of this kind are the following: Subgraph Isomorphism (given graphs $H$ and $G$, check whether $H$ is a subgraph of $G$), Distance-$r$ Independent Set (given a graph $G$, find as many as possible
vertices pairwise at distance more than $r$), and Distance-$r$ Dominating Set (given a graph $G$, use as few vertices as possible to dominate it, 
where every vertex dominates all vertices at distance at most $r$ from it).
The study of these particular problems in the context of structural sparsity inspired a range of new algorithmic techniques, based on different combinatorial tools.

The notion of {\em{low tree-depth decompositions}} introduced by Ne\v{s}et\v{r}il and Ossona de Mendez~\cite{NesetrilM08a} was directly inspired by algorithmic layering techniques known from the planar setting,
and provides alternative combinatorial characterizations of bounded expansion and nowhere denseness.
One of the first applications, observed in~\cite{NesetrilM08a}, is a linear-time fixed-parameter algorithm for Subgraph Isomorphism on any class of bounded expansion.
%which provides the base, existential case for first-order model-checking. 
The fixed-parameter tractability of Distance-$r$ Dominating Set on any nowhere dense class was proved by Dawar and Kreutzer~\cite{DawarK09}, 
who used the characterization of nowhere denseness via {\em{uniform quasi-wideness}} to this aim.
This line of research was continued by Drange et al.~\cite{DrangeDFKLPPRVS16} and by Eickmeyer et al.~\cite{eickmeyer2016neighborhood}, 
who gave a linear kernel for the problem on any class of bounded expansion and an almost linear kernel on any nowhere dense class, respectively.
These works highlighted the combinatorial concept of {\em{neighborhood complexity}} as a useful way of defining structural sparsity, 
which was later generalized and connected to concepts from stability theory by Pilipczuk et al.~\cite{pilipczuk2018number}.
It turns out that for subgraph-closed classes the notion of nowhere denseness is the ultimate limit of fixed-parameter tractability for Distance-$r$ Dominating Set,
under plausible complexity assumptions~\cite{DrangeDFKLPPRVS16}. This, together with a similar dichotomy result for model-checking first-order logic that we discuss later,
validates the fundamental nature of the notion of nowhere denseness: it is the natural of tractability for local problems on subgraph-closed classes of graphs.

The study of duality between distance-$2r$ independence and distance-$r$ domination led Dvo\v{r}\'ak~\cite{Dvorak13}
to the development of constant-factor approximation algorithms for both problems on any class of bounded expansion.
The algorithm of Dvo\v{r}\'ak is remarkably simple and it relies on {\em{generalized coloring numbers}}, which influenced a widespread usage of this approach in algorithmic aspects of sparsity.
This applies in particular to the setting of {\em{distributed algorithms}}, because using generalized coloring numbers can be used to compute sparse {\em{neighborhood covers}}: 
a covering of a graph using local clusters that enables localization of computation and, thus, facilitates distributed treatment.
Using these ideas, Amiri et al.~\cite{AmiriMRS18} gave a distributed constant factor approximation algorithm for Distance-$r$ Dominating Set on any class of bounded expansion running in a logarithmic number of rounds.
A distributed variant of the algorithm for Subgraph Isomorphism running in a logarithmic number of rounds was presented by Ne\v{s}et\v{r}il and Ossona de Mendez~\cite{NesetrilM16}.

Recently, sparsity methods were successfully applied to give new algorithmic results in the settings of {\em{circuit complexity}}~\cite{PilipczukST18} 
(a model for {\em{parallel algorithms}}), {\em{property testing}}~\cite{AdlerH18}, and {\em{approximation schemes}}~\cite{Har-PeledQ17}.



\paragraph*{Eyes of an applied computer scientist.}

\paragraph*{Expected outcomes.}
The main goal of the proposed seminar is to create a platform for the exchange of tools, ideas, and questions between researchers working on different aspects of the theory.
So far, such synergy led to major developments due to the multi-disciplinary character of the field, and we hope that the seminar will foster new cross-field collaborations and lead to new results.
Besides this, the studied concepts of sparsity are very young and the new techniques are still not widely known in the related fields. 
This particularly applies to the field of algorithm design, where the toolbox seems to be applicable within multiple paradigms, whose respective communities are not aware of the new developments.
It would be desired if the seminar contributed to the visibility of the theory of sparse graph classes within neighboring areas and inspired new results based on its techniques.
Finally, the recent attempts at deploying theoretical sparsity-based methods in practice show promise and we believe that the community should strongly support applications-driven research.
We hope that the seminar will encourage more practical works such as implementation, evaluation, and fine-tuning of theoretical methods, and the usage of sparsity-based subroutines in larger systems.

\paragraph*{Structure of the seminar.}
We intend to keep the seminar oriented on research and collaboration.
A block of talks will be scheduled each morning, while afternoons will be kept free for collaboration.
The talks will be a mix of invited tutorials on important techniques,
keynote lectures on recent developments, and shorter contributed talks.
We intend to prioritize early career researchers for giving contributed talks and
we will make sure that they take an active part in the afternoon discussions,
including encouraging them to organize working groups on topics discussed during the workshop.
To guarantee the collaborative atmosphere and to assist with identifying hot topic problems to be discussed,
we plan to organize an open problems session in the early afternoon of the first day.
To facilitate a smooth and lively interaction during the problem session,
we will contact several key participants beforehand and ask them to prepare open problems to present.
If needed, a second open problem session or a progress report session will be held later during the week.

\paragraph*{Other seminars, workshops, and projects.}
We are aware of the following recent events connected to the topic of the proposed seminar.
\begin{itemize}
\item A workshop on structural sparsity, logic, and algorithms was organized by Anuj Dawar, Zdenek Dvo\v{r}\'ak, and Daniel Kr\'al' at the University of Warwick in June 2018. 
It was a follow-up of a workshop on similar topics organized at the same place in December 2016 by Daniel Kr\'al' and Ranko Lazi\'c. See \url{https://warwick.ac.uk/fac/sci/maths/people/staff/daniel_kral/strlogalg/}
and \url{https://warwick.ac.uk/fac/sci/maths/people/staff/daniel_kral/alglogstr}.
\item In Autumn 2018, Charles University in Prague is organizing DOCCOURSE: a 1.5-month long program on structural sparsity in connection with model theory
for senior undergraduate and graduate students. See \url{https://iuuk.mff.cuni.cz/events/doccourse2018/}.
\item A one-day satellite workshop on algorithms and structure for sparse graphs was organized during the 44th International Colloquium on Automata, Languages, and Programming, ICALP 2017, held in Warsaw in July 2017.
\item TODO: Paris event
\end{itemize}
We are not aware of any previous Dagstuhl seminars directly overlapping with the subject of our proposal.

\section{Invitee list}

\invitee{Isolde Adler}{University of Leeds}{UK}{I.M.Adler@leeds.ac.uk}{https://engineering.leeds.ac.uk/staff/810/Dr_Isolde_Adler}{Algorithms, logic}{female}

\invitee{Marthe Bonamy}{LaBRI Bordeaux and CNRS}{France}{marthe.bonamy@u-bordeaux.fr}{http://www.labri.fr/perso/mbonamy/}{Combinatorics, algorithms}{female junior}

\invitee{Nicolas Bousquet}{University Grenoble Alpes and CNRS}{France}{nicolas.bousquet@grenoble-inp.fr}{https://pagesperso.g-scop.grenoble-inp.fr/~bousquen/}{Combinatorics, algorithms}{junior}

\invitee{Anuj Dawar}{University of Cambridge}{UK}{anuj.dawar@cl.cam.ac.uk}{https://www.cl.cam.ac.uk/~ad260/}{Combinatorics, algorithms, logic}{}

\invitee{Reinhard Diestel}{University of Hamburg}{Germany}{ReinhardDiestel@math.uni-hamburg.de}{https://www.math.uni-hamburg.de/home/diestel/}{Combinatorics, algorithms}{}

\invitee{Zdeněk Dvořák}{Charles University, Prague}{Czechia}{ook@ucw.cz}{https://iuuk.mff.cuni.cz/~rakdver/}{Combinatorics, algorithms, logic}{}

\invitee{Eduard Eiben}{University of Bergen}{Norway}{Eduard.Eiben@uib.no}{https://www.uib.no/en/persons/Eduard.Eiben}{Algorithms}{junior}

\invitee{Kord Eickmeyer}{Technical University, Darmstadt}{Germany}{eickmeyer@mathematik.tu-darmstadt.de}{https://www2.mathematik.tu-darmstadt.de/~eickmeyer/}{Algorithms, logic}{}

\invitee{Fedor Fomin}{University of Bergen}{Norway}{fomin@ii.uib.no }{http://www.ii.uib.no/~fomin/}{Algorithms}{}

\invitee{Jakub Gajarský}{Technical University, Berlin}{Germany}{jakub.gajarsky@tu-berlin.de}{http://logic.las.tu-berlin.de/Members/Gajarsky/}{Algorithms, logic}{junior}

\invitee{Robert Ganian}{Technical University, Vienna}{Austria}{rganian@ac.tuwien.ac.at}{https://www.ac.tuwien.ac.at/people/rganian/}{Algorithms, logic}{junior}

\invitee{Archontia Giannopoulou}{Technical University, Berlin}{Germany}{Archontia.Giannopoulou@gmail.com}{http://users.uoa.gr/~arcgian/index.html}{Combinatorics, algorithms}{female junior}

\invitee{Martin Grohe}{RWTH Aachen}{Germany}{grohe@informatik.rwth-aachen.de }{}{Combinatorics, algorithms, logic}{}

\invitee{Sariel Har-Peled}{University of Illinois in Urbana-Champaign}{USA}{sariel@uiuc.edu}{https://sarielhp.org/}{Algorithms}{}

\invitee{Petr Hliněný}{Masaryk University, Brno}{Czechia}{hlineny@fi.muni.cz}{https://www.fi.muni.cz/~hlineny/}{Algorithms, logic}{}

\invitee{Gwenaël Joret}{Free University of Brussels}{Belgium}{gjoret@ulb.ac.be}{http://di.ulb.ac.be/algo/gjoret/}{Combinatorics}{}

\invitee{Ken-ichi Kawarabayashi}{National Institute of Informatics, Tokyo}{Japan}{kkeniti@nii.ac.jp}{}{Combinatorics, algorithms}{}

\invitee{Tereza Klimošová}{Charles University, Prague}{Czechia}{tereza@kam.mff.cuni.cz}{https://iuuk.mff.cuni.cz/~tereza/}{Combinatorics}{female junior}

\invitee{Daniel Krá\v{l}}{Masaryk University, Brno}{Czechia}{dkral@fi.muni.cz}{http://www.ucw.cz/~kral/}{Combinatorics, algorithms, logic}{}

\invitee{Stephan Kreutzer}{Technical University, Berlin}{Germany}{stephan.kreutzer@tu-berlin.de}{http://logic.las.tu-berlin.de/Members/Kreutzer/}{Algorithms, logic}{}

\invitee{O-joung Kwon}{Incheon National University}{South Korea}{ojoungkwon@inu.ac.kr}{http://ojkwon.com/}{Algorithms}{junior}

\invitee{Aurelie Lagoutte}{University Clermont-Auvergne}{France}{aurelie.lagoutte@uca.fr}{https://fc.isima.fr/~alagoutte/}{Combinatorics}{female junior}

\invitee{Daniel Lokshtanov}{University of California in Santa Barbara}{USA}{daniello@ucsb.edu}{https://www.cs.ucsb.edu/people/faculty/lokshtanov}{Algorithms}{}

\invitee{Dániel Marx}{Hungarian Academy of Sciences (MTA SZTAKI)}{Hungary}{dmarx@cs.bme.hu }{http://www.cs.bme.hu/~dmarx/}{Combinatorics, algorithms}{}

\invitee{Claire Mathieu}{CNRS}{France}{cmathieu@di.ens.fr}{https://www.di.ens.fr/ClaireMathieu.html}{Algorithms}{}

\invitee{Piotr Micek}{Jagiellonian University, Cracow}{Poland}{piotr.micek@tcs.uj.edu.pl}{http://www.tcs.uj.edu.pl/micek}{Combinatorics}{}

\invitee{Amer Mouawad}{University of Bergen}{Norway}{amer.mouawad@gmail.com }{https://folk.uib.no/amo110/}{Algorithms}{junior}

\invitee{Irene Muzi}{University of Warsaw}{Poland}{irene.muzi@gmail.com}{https://sites.google.com/site/irenemuzigraph/home}{Combinatorics, algorithms}{female junior}

\invitee{Wojciech Nadara}{University of Warsaw}{Poland}{wojtek.nadara@gmail.com }{}{Algorithms, applications}{junior}

\invitee{Jaroslav Nešetřil}{Charles University, Prague}{Czechia}{nesetril@iuuk.mff.cuni.cz}{https://iuuk.mff.cuni.cz/~nesetril/en/}{Combinatorics, algorithms, logic}{}

\invitee{Jan Obdržálek}{Masaryk University, Brno}{Czechia}{obdrzalek@fi.muni.cz}{https://www.fi.muni.cz/~xobdrzal/}{Algorithms, logic}{}

\invitee{Patrice Ossona de Mendez}{CAMS Paris, CNRS, and Charles University, Prague}{France/Czechia}{patrice.ossona-de-mendez@ehess.fr }{http://cams.ehess.fr/patrice-ossona-de-mendez/}{Combinatorics, algorithms, logic}{}

\invitee{Sang-il Oum}{KAIST, Daejeon}{South Korea}{sangil@kaist.edu}{https://mathsci.kaist.ac.kr/~sangil/}{Combinatorics, algorithms}{}

\invitee{Fahad Panolan}{University of Bergen}{Norway}{Fahad.Panolan@uib.no}{https://folk.uib.no/fpa082/}{Algorithms}{junior}

\invitee{Marcin Pilipczuk}{University of Warsaw}{Poland}{marcin.pilipczuk@mimuw.edu.pl}{https://www.mimuw.edu.pl/~malcin/}{Algorithms, applications}{}

\invitee{Michał Pilipczuk}{University of Warsaw}{Poland}{michal.pilipczuk@mimuw.edu.pl}{https://www.mimuw.edu.pl/~mp248287/}{Algorithms, logic}{}

\invitee{Kent Quanrud}{University of Illinois in Urbana-Champaign}{USA}{quanrud2@illinois.edu}{http://quanrud2.web.engr.illinois.edu/}{Algorithms}{junior}

\invitee{Daniel Quiroz}{University of Chile, Santiago de Chile}{Chile}{dquiroz@cmm.uchile.cl}{http://www.cmm.uchile.cl/~dquiroz/}{Combinatorics}{junior}

\invitee{Roman Rabinovich}{Technical University, Berlin}{Germany}{roman.rabinovich@tu-berlin.de}{http://logic.las.tu-berlin.de/Members/Rabinovich/}{Algorithms, applications}{junior}

\invitee{Jean-Florent Raymond}{Technical University, Berlin}{Germany}{raymond@tu-berlin.de}{http://www.user.tu-berlin.de/jraymond/}{Combinatorics, algorithms}{junior}

\invitee{Felix Reidl}{Birbeck University of London}{UK}{f.reidl@dcs.bbk.ac.uk}{https://tcs.rwth-aachen.de/~reidl/}{Combinatorics, algorithms}{junior}

\invitee{Saket Saurabh}{Institute of Mathematical Sciences, Chennai}{India}{saket@imsc.res.in}{http://www.ii.uib.no/~saket/}{Algorithms}{}

\invitee{Nicole Schweikardt}{Humboldt University, Berlin}{Germany}{schweikn@informatik.hu-berlin.de}{https://www2.informatik.hu-berlin.de/~schweikn/}{Algorithms, logic}{female}

\invitee{Luc Segoufin}{INRIA and ENS Ulm}{France}{luc.segoufin@inria.fr}{https://who.rocq.inria.fr/Luc.Segoufin/}{Algorithms, logic}{}

\invitee{Sebastian Siebertz}{Humboldt University, Berlin}{Germany}{siebertz@informatik.hu-berlin.de}{https://www.informatik.hu-berlin.de/de/institut/mitarbeiter/1691486}{Algorithms, logic}{junior}

\invitee{Konstantinos Stavropoulos}{University of Hamburg}{Germany}{konstantinos.stavropoulos@uni-hamburg.de}{https://www.math.uni-hamburg.de/home/stavropoulos/}{Combinatorics}{junior}

\invitee{Blair Sullivan}{North Carolina State University, Raleigh}{USA}{blair_sullivan@ncsu.edu}{https://people.engr.ncsu.edu/vbsulliv/}{Algorithms, applications}{female}

\invitee{Dimitrios Thilikos}{LIRMM Montpellier and CNRS}{France}{sedthilk@thilikos.info}{https://www.lirmm.fr/~thilikosto/}{Combinatorics, algorithms}{}

\invitee{Szymon Toruńczyk}{University of Warsaw}{Poland}{szymtor@mimuw.edu.pl}{https://www.mimuw.edu.pl/~szymtor/}{Algorithms, logic}{}

\invitee{Torsten Ueckerdt}{Karlsruhe Institute of Technology}{Germany}{torsten.ueckerdt@kit.edu}{https://i11www.iti.kit.edu/en/members/torsten_ueckerdt/index}{Combinatorics}{junior}

\invitee{Jan van den Heuvel}{London School of Economics and Political Science}{UK}{j.van-den-heuvel@lse.ac.uk}{http://www.maths.lse.ac.uk/personal/jan/}{Combinatorics}{}

\invitee{Alexandre Vigny}{University of Warsaw}{Poland}{alexandre.vigny@imj-prg.fr}{https://webusers.imj-prg.fr/~alexandre.vigny/}{Algorithms, logic}{junior}

\invitee{Kristina Vušković}{University of Leeds}{UK}{K.Vuskovic@leeds.ac.uk}{https://engineering.leeds.ac.uk/staff/249/kristina_vuskovic}{Combinatorics}{female}

\invitee{Bartosz Walczak}{Jagiellonian University, Cracow}{Poland}{walczak@tcs.uj.edu.pl}{http://www.tcs.uj.edu.pl/walczak}{Combinatorics}{}

\invitee{Daniel Weissauer}{University of Hamburg}{Germany}{daniel.weissauer@uni-hamburg.de}{}{Combinatorics}{junior}

\invitee{Paul Wollan}{University of Rome “La Sapienza”}{Italy}{wollan@di.uniroma1.it}{http://wwwusers.di.uniroma1.it/~wollan/}{Combinatorics, algorithms}{}

\invitee{David Wood}{Monash University, Melbourne}{Australia}{david.wood@monash.edu}{http://users.monash.edu.au/~davidwo/}{Combinatorics, algorithms}{}

\invitee{Marcin Wrochna}{University of Oxford}{UK}{m.wrochna@mimuw.edu.pl }{https://www.mimuw.edu.pl/~mw290715/}{Combinatorics, algorithms}{junior}

\invitee{Meirav Zehavi}{Ben-Gurion University}{Israel}{zehavimeirav@gmail.com}{https://sites.google.com/site/zehavimeirav/}{Algorithms}{female junior}

\invitee{Xuding Zhu}{National Sun Yat-sen University, Kaohsiung}{Taiwan}{zhu@math.nsysu.edu.tw}{http://www.math.nsysu.edu.tw/~zhu/}{Combinatorics}{}



\section{Organizers' CVs}

\section{Proposed dates}

We would be most happy to organize the seminar either in Autumn 2019 or in Spring 2020. The following blocks of dates would be most suitable:
\begin{itemize}
\item 30.09.2019 --- 15.11.2019;
\item 02.03.2020 --- 03.04.2020;
\item 20.04.2020 --- 24.04.2020.
\end{itemize}
As for block-out dates, the following weeks should be excluded from consideration:
\begin{itemize}
\item 10.09.2019 --- 14.09.2019.
\end{itemize}


\bibliographystyle{abbrv}
\bibliography{references}



\end{document}
