\documentclass[10pt]{article}

\usepackage{fullpage}
\usepackage[all=normal,bibliography=tight]{savetrees}
\usepackage[left=1in,right=1in,top=0.8in,bottom=0.8in]{geometry}
            
\usepackage{amsmath,amsthm,amssymb}
\usepackage{hyperref} 
\usepackage{todonotes}
\usepackage{xspace}
\usepackage{ifthen}
\usepackage[czech, english]{babel}
\selectlanguage{english} % small trick to get $\v{l} for Dan ....

\usepackage[utf8]{inputenc}
\usepackage{comment}

% ====================


\newcommand{\ignore}[1]{}
\newcommand{\CC}{{\bf{C:}}\quad}
\newcommand{\JJ}{{\bf{J :}}\quad}

\newcommand{\heading}[1]{
  \vfill
  \vspace{0.2cm}
  \hrule height .2mm  \vspace{1mm}
  \noindent {\large \textbf{#1}}\\ \vspace{-3mm} \hrule height .2mm
  \vspace{0.3cm}
}

\newcommand{\smallheading}[1]{
  \vfill
  \vspace{0.2cm}
  \hrule height .2mm  \vspace{1mm}
  \noindent {\small \textbf{#1}}\\ \vspace{-3mm} \hrule height .2mm
  \vspace{0.3cm}
}

%\addtolength{\hoffset}{-28mm}
%\addtolength{\textwidth}{56mm}
%\addtolength{\voffset}{-20mm}
%\addtolength{\textheight}{28mm}

%\hyphenpenalty=100000

%\pagenumbering{none}

\newcounter{pcount}

\newcommand{\paperplain}[2]{
\noindent\begin{tabular}{@{}p{0.7cm} @{\hspace{2mm}} p{16.8cm}}
[\arabic{pcount}\refstepcounter{pcount}\label{#1}] & #2
\end{tabular}
\vskip 0.2cm
}

\newcommand{\paper}[4]{
\noindent\begin{tabular}{@{}p{0.7cm} @{\hspace{2mm}} p{15.3cm}}
[\arabic{pcount}\refstepcounter{pcount}\label{#1}] & #2, {\em{#3}},\\ & #4.
\end{tabular}
\vskip 0.2cm
}
\newcommand{\paperbr}[4]{
\noindent\begin{tabular}{@{}p{0.7cm} @{\hspace{2mm}} p{15.3cm}}
[\arabic{pcount}\refstepcounter{pcount}\label{#1}] & #2, \\ & {\em{#3}},\\ & #4.
\end{tabular}
\vskip 0.2cm
}

\newcommand{\paperlncs}[5]{
\noindent\begin{tabular}{@{}p{0.7cm} @{\hspace{2mm}} p{16.8cm}}
[\arabic{pcount}\refstepcounter{pcount}\label{#1}] & #2, \\
& {\em{#3}},\\
& #4 \\
& #5
\end{tabular}
\vskip 0.2cm
}


\newcommand{\paperk}[5]{
\noindent\begin{tabular}{@{}p{0.7cm} @{\hspace{2mm}} p{16.8cm}}
[\arabic{pcount}\refstepcounter{pcount}\label{#1}] & #2, \\
& {\em{#3}},\\
& #4 \\[0.1cm]
& {\bf{Conf.}}: #5
\end{tabular}
\vskip 0.2cm
}

\newcommand{\paperklncs}[6]{
\noindent\begin{tabular}{@{}p{0.7cm} @{\hspace{2mm}} p{16.8cm}}
[\arabic{pcount}\refstepcounter{pcount}\label{#1}] & #2, \\
& {\em{#3}},\\
& #4 \\[0.1cm]
& {\bf{Conf.}}: #5\\
& \hskip 0.96cm #6
\end{tabular}
\vskip 0.2cm
}

\newcommand{\papercm}[5]{
\noindent\begin{tabular}{@{}p{0.7cm} @{\hspace{2mm}} p{16.8cm}}
[\arabic{pcount}\refstepcounter{pcount}\label{#1}] & #2, \\
& {\em{#3}},\\
& #4 \\[0.1cm]
& {\bf{Note}}: #5
\end{tabular}
\vskip 0.2cm
}

\newcommand{\papercmlncs}[6]{
\noindent\begin{tabular}{@{}p{0.7cm} @{\hspace{2mm}} p{16.8cm}}
[\arabic{pcount}\refstepcounter{pcount}\label{#1}] & #2, \\
& {\em{#3}},\\
& #4 \\
& #5 \\[0.1cm]
& {\bf{Note}}: #6
\end{tabular}
\vskip 0.2cm
}

\newcommand{\papercmk}[6]{
\noindent\begin{tabular}{@{}p{0.7cm} @{\hspace{2mm}} p{16.8cm}}
[\arabic{pcount}\refstepcounter{pcount}\label{#1}] & #2, \\
& {\em{#3}},\\
& #4 \\[0.1cm]
& {\bf{Conf.}}: #6\\
& {\bf{Note}}: #5
\end{tabular}
\vskip 0.2cm
}

\newcommand{\papercmklncs}[7]{
\noindent\begin{tabular}{@{}p{0.7cm} @{\hspace{2mm}} p{16.8cm}}
[\arabic{pcount}\refstepcounter{pcount}\label{#1}] & #2, \\
& {\em{#3}},\\
& #4 \\[0.1cm]
& {\bf{Conf.}}: #6\\
& \hskip 0.96cm #7\\
& {\bf{Note}}: #5
\end{tabular}
\vskip 0.2cm
}


\newcommand{\software}[5]{
\noindent\begin{tabular}{@{}p{0.7cm} @{\hspace{2mm}} p{15.3cm}}
[\arabic{pcount}\refstepcounter{pcount}\label{#1}] & #2, \\
& {\textsf{#3}}, #4 \\
& {\footnotesize #5}
\end{tabular}
\vskip 0.2cm
}


% =========================


\newcommand{\ourtitle}{Sparsity in Algorithms, Combinatorics and Logic}
\newcommand{\email}[1]{\href{mailto:#1}{\nolinkurl{#1}}}
\newcommand{\organizer}[7]{%
\noindent\begin{tabular}{p{3cm}l}
\textbf{Name:} & #1 \\
\textbf{Affiliation:} & #2 \\
\textbf{Address:} & #3 \\
\textbf{Phone:} & #4 \\
\textbf{Fax:} & #5 \\
\textbf{Email:} & \email{#6} \\
\textbf{Homepage:} & \url{#7}
\end{tabular}\par\vspace{4mm}}

\newcommand{\invitee}[7]{%
\noindent\begin{tabular}{p{3cm}l}
\textbf{Name:} & #1 \\
\textbf{Affiliation:} & #2 \\
\textbf{Country:} & #3 \\
\textbf{Email:} & \email{#4} \\
\ifthenelse{\equal{#5}{}}{}{\textbf{Homepage:} & \url{#5} \\}
\textbf{Topics:} & #6
\ifthenelse{\equal{#7}{}}{}{\\ \textbf{Tags:} & #7}
\end{tabular}\par\vspace{2mm}}

\renewcommand\thesection{\Alph{section}}

\newcommand{\Cc}{\mathcal{C}}

\title{{\large{Proposal for a Dagstuhl seminar on}}\\
  {\LARGE{\textbf{\ourtitle}}}}

\author{
  Daniel Kr\'a\v{l} \and
  Micha\l{} Pilipczuk \and
  Sebastian Siebertz\and
  Blair D. Sullivan 
}

\date{}
%\date{\today}

\begin{document}
\maketitle

\begin{abstract}
The concept of \emph{sparsity} in combinatorics
aims at capturing abstract notions of uniform sparsity of graphs,
as well as more general relational structures.
The main goal is to obtain understanding why, and to what extent,
structural properties related to the sparsity of a given structure
can be used to develop sparsity-based methods and algorithms.
Since the pioneering work of Ne\v{s}et\v{r}il and Ossona de Mendez from the late 2000's,
a huge body of research has shown that the two main sparsity concepts --- \emph{bounded expansion} and \emph{nowhere denseness} ---
have deep connections to classic notions from algorithm design, combinatorics and model theory, and
they can be used to obtain powerful structural and algorithmic results.
It is the synergy of algorithm design, combinatorics and model theory that
makes studying sparsity mathematically exciting and brings so many computer science applications.
One of the many evidences of the fundamental nature of the concepts of bounded expansion and nowhere denseness is that
these two concepts constitute the boundary of the computational tractability for many natural classes of problems, 
most notably first-order model checking.

The aim of the proposed seminar is to bring together researchers working on various aspects of sparsity in their own fields,
in order to facilitate the exchange of ideas, methods and questions between different communities.
An important part of the seminar will be the discussion of the (still fledgling) area of real-life applications of
sparsity-based methods, where theory and practice could meet.
%The theory of {\em{sparsity}} studies abstract notions of uniform sparsity for classes of graphs, as well as more general logical structures.
%The main goal of the theory is to understand why, and to what extent, sparsity of a given structure can be used to describe its properties, 
%and to develop tools helpful for designing sparsity-based methods.
%Since the work of Ne\v{s}et\v{r}il and Ossona de Mendez, who laid solid foundations of the theory in the late 2000's, 
%a huge body of work has shown that the two main notions --- {\em{bounded expansion}} and {\em{nowhere denseness}} ---
%have deep connections with classic concepts from combinatorics, model theory, and algorithm design, and can be used to obtain new, powerful results in these areas.
%It is the synergy of these three fields that makes sparsity a mathematically rich and exciting theory, which is currently under rapid development.
%The notions of bounded expansion and nowhere denseness also constitute borders of computational tractability for natural classes of problems, 
%most notably for the model-checking problem for first-order logic; this witnesses the fundamental nature of the studied concepts.
%
%The aim of the proposed seminar is to bring together researchers working on various aspects of sparsity, in order to facilitate 
%the exchange of ideas, methods, and questions between different communities. An important part of the seminar will be the discussion
%of the (still fledgling) area of real-life applications of sparsity-based methods, where theory and practice could meet.
\end{abstract}

\section{Metadata}

\begin{enumerate}
\item {\bf{Organizers}}:
\begin{itemize}
\item Daniel Kr\'a\v{l}, Masaryk University in Brno, Czech Republic, \email{dkral@fi.muni.cz};
\item Micha\l{} Pilipczuk, University of Warsaw, Poland, \email{michal.pilipczuk@mimuw.edu.pl};
\item Sebastian Siebertz, Humboldt-Universit\"at zu Berlin, Germany, \mbox{\email{siebertz@informatik.hu-berlin.de};}
\item Blair D. Sullivan, North Carolina State University, Raleigh, USA, \email{blair_sullivan@ncsu.edu}.
\end{itemize}
\item {\bf{Title}}: \ourtitle
\item {\bf{Type}}: Dagstuhl Seminar
\item {\bf{Size and duration}}: Large (45 participants) and Long (5 days)
\item {\bf{Classification}}: data structures, algorithms, complexity
\item {\bf{Keywords}}: Algorithm design, parameterized complexity, sparse graphs, graph decompositions, model theory.
\end{enumerate}

\pagebreak
\section{Proposal text}

%Blah blah blah~\cite{sparsity}

\paragraph*{Introduction.}
It was realized already in the early days of computer science that structures (networks, databases, etc.) that are {\em{sparse}} appear ubiquitously in applications.
The sparsity of input can be used in a variety of ways, e.g. to design efficient algorithms. This motivates a theoretical study of the abilities and limitations of sparsity-based methods.
However, a priori it is not clear how to even define sparsity formally.
Focusing on graphs with bounded maximum degree seems too restrictive, as real-life networks tend to have high-degree hubs while retaining a moderate overall number of edges,
while only requiring a constant upper bound on the average degree is too relaxed, as it allows the existence of dense substructures.
Another way is to concentrate on classes with topological constraints, e.g. planar graphs or, more generally, proper minor-closed classes,
or to generalize the example of bounded-degree networks to classes with bounded {\em{degeneracy}} (equivalently, bounded {\em{arboricity}}).
While these ideas inspired multiple results, it can be argued that they did not lead to the development of a robust mathematical theory.

In the late 2000s, Ne\v{s}et\v{r}il and Ossona de Mendez~\cite{NesetrilM08,NesetrilM08a,NesetrilM08b} proposed a different approach:
to introduce new definitions of uniform, structural sparsity for classes of graphs that would generalize the currently considered settings,
and to develop a toolbox of sparsity-based methods for such sparse classes.
The central notions they introduced are classes of {\em{bounded expansion}} and {\em{nowhere dense}} classes.
In a nutshell, a class of graphs $\Cc$ has bounded expansion if by contracting constant-radius connected subgraphs into single vertices
in any graph from $\Cc$ one cannot obtain graphs with arbitrary high average degree; for nowhere denseness, we only insist that in this way one cannot obtain arbitrary large cliques.
Thus, bounded expansion and nowhere denseness can be thought of uniform sparsity that persists under local modifications, modelled by constant-radius contractions.
Every class of bounded expansion is nowhere dense, and every class that excludes a fixed topological minor --- in particular every proper minor-closed class and every class of bounded maximum degree ---
has bounded expansion. Thus, these concepts are indeed broader than the previously considered settings, while sparse networks appearing in applications indeed tend to come from
classes of bounded expansion~\cite{DemaineRRVSS14}.

It quickly turned out that the proposed notions can be used to build a mathematical theory of sparse graph classes that offers a wealth of tools, leading to new techniques and powerful results.
Indeed, the last~10 years have seen a great progress in the area, witnessed by many publications in top venues.
It is particularly remarkable that the concepts of bounded expansion and nowhere denseness can be connected
to fundamental ideas from multiple other fields of computer science, often in a surprising way, and thus there are several complementary viewpoints on the subject.
On one hand, foundations of the area are grounded in {\em{structural graph theory}}, which aims at describing structure in graphs through various decompositions and auxiliary parameters.
On the other hand, nowhere denseness seems to delimit the border of expressibility and algorithmic tractability of first-order logic, which provides a link to {\em{finite model theory}}.
Finally, there is a fruitful transfer of ideas to and from the field {\em{algorithm design}}: sparsity-based methods can be used to design new, efficient algorithms,
and classic techniques of designing algorithms on sparse inputs inspire new combinatorial results on sparse graphs.
It is the synergy of these three fields that makes the theory of sparse graphs so exciting.
In the next paragraphs we explain how the recent progress can be viewed through the eyes of researchers working in these areas.

The aim of the proposed seminar is to ``stir in the pot'' and invite researchers working with structurally sparse structures to share their experience, questions, and ideas.
An important secondary goal is to facilitate discussion on practical applications of (so far) theoretical methods that were recently developed.
%We believe that the recent rapid progress in the area calls for capitalizing on the momentum.

\paragraph*{Eyes of a combinatorialist.}

%Graph classes with restricted structure (such as minor closed classes) and classes based on various notions of width
%play an important role in algorithm design.  Classical results from the theory of graph minors have recently been
%extended to more general classes of graphs, e.g. those avoiding immersions and subdivisions, which provided new insights
%on their structure including the existence of small-size separators.  The introduction of graph classes with bounded
%expansion and nowhere-dense classes has led to a new robust way of understanding sparsity with many implications
%in combinatorics and computer science.  An important direction seeks to transfer methods from the study of sparse
%classes to non-sparse classes, such as graphs of bounded clique-width and their generalisations.

\paragraph*{Algorithms.}
\paragraph*{Eyes of a model theorist.}
Model theory  is the study of classes of mathematical structures from the perspective of mathematical logic, and this study mainly concerns infinite structures.
A classification of theories based on those whose models can be classified and those whose models are too complicated to classify has been initiated by Shelah, who  drew two important dividing lines: NIP vs dependent and stable vs unstable \cite{shelah1990classification}.

Finite model theory, which focuses on finite structures,  diverges significantly from the study of infinite structures in both the problems studied and the techniques used, and mainly grew out of computer science applications.  This led to the development of strong tools to study logics over finite structures, which helped to answer many questions about  complexity theory, databases, formal languages, algorithms, and many more. This also led to the notion of {\em tame} classes of finite structures, which behave well.  In this line Dawar \cite{Dawar2010} introduced the notion of a \emph{uniform quasi-wide class} and proved 
     that the homomorphism preservation theorem holds when relativized to a uniformly quasi-wide class of finite structures. It was later shown 
by Ne\v{s}et\v{r}il and Ossona de Mendez  that the uniform quasi-wide classes are exactly nowhere dense classes \cite{ND_logic}.

The  structural and algorithmic properties of  nowhere dense classes  are deeply connected to  the model theory notions of independence and stability, as witnessed in particular by the following two fundamental results. The first asserts that for a monotone class of graphs, the notions of nowhere dense class, NIP class, and stable class are equivalent \cite{adler2014interpreting}. The second states that nowhere dense classes are the most general monotone classes for which first-order model checking is fixed-parameter tractable \cite{grohe2017deciding}. This connection has already proved its efficiency in transferring techniques and notions from model theory to combinatorics (see for instance \cite{pilipczuk2018number}) and, conversely, from combinatorics to finite model theory (see for instance \cite{rossman2008homomorphism}).

Numerous algorithmic applications of structural properties of the graphs in a nowhere dense class (or, more restrictively in a bounded expansion class) have appeared, including   a near-optimal kernelization algorithm for the distance-$r$ dominating set problem for the graphs in a nowhere dense class \cite{eickmeyer2016neighborhood}.

Guided by the model theory intuition, it is natural to expect that some of the techniques and results proved for monotone sparse classes may extend to hereditary structurally sparse classes, that is possibly dense classes of low structural complexity. In this direction, it has been proved that model checking is fixed parameter tractable on map graphs \cite{eickmeyer2017fo} and on interpretations of classes of graphs with bounded maximum degree \cite{gajarsky2016new} and that transductions of bounded expansion classes of graphs admit low shrub-depth decompositions, a dense analog of low tree-depth decompositions \cite{gajarsky2018first}. 


\paragraph*{Applied eyes of an algorithm designer.}

Over the last twenty years, the field of network science
has burgeoned, developing new methods for complex network data arising
in diverse fields including social networks, bioinformatics, quantum computing,
transportation, and healthcare. Surprisingly, few tools from structural graph theory have been assimilated into
their arsenal. In part, this is due to the theoretical nature of
much of the related literature on parameterized graph algorithms
and a lack of cross-pollination of the research
communities.

There has been some work on making structure-based approaches more practical, mostly focused on
bounded treewidth and planar graphs. When working with bounded treewidth, one significant
challenge is finding low-width tree decompositions -- here, work by Bodlaender and Koster (e.g.~\cite{bodlaender2006-tw, koster2001-tw})
set the stage for the PACE challenge (https://pacechallenge.org). More recently, there have also been
experimental evaluations of treewidth-based algorithms for solving downstream optimization problems like
Vertex Cover and Hamiltonian Cycle~\cite{ziobro2018hamcycle-tw, alber2005vc-tw}. In an even more
applied setting, recently treewidth solvers were shown to be competitive for finding contraction orderings
for tensor networks used in quantum computing simulations~\cite{dumitrescu2018tensors}. There has also been some
work on parallel algorithms for bounded treewidth graphs, including a distributed memory implementation of a
Maximum Weighted Independent Set solver~\cite{sullivan2013paralleltd}.
Planar graphs have also attracted more practical interest, in part due to their natural representation of problems with geographical
constraints (e.g.~\cite{alber2001-planar,schmidt2009-planarvision}).

Since the introduction of structural sparsity and bounded expansion,
there has been a resurgence in interest, in part because of work showing that
this class includes many real-world networks (e.g.~\cite{DemaineRRVSS14}).
Similar to algorithms for bounded treewidth, pipelines for structurally sparse graph classes
require computation of a ``decomposition,'' such as a low-treedepth coloring. An end-to-end implementation
of the algorithm for subgraph isomorphism counting~\cite{obrien2017concuss} revealed that the
number of colors used in the current constant-factor approximation algorithms was causing a significant
practical bottleneck. Subsequently, there has been work on comparing algorithms and heuristics
for computing these parameters~\cite{wojciech2018quasiwide}, as well as more theoretical research
on alternative colorings that trade off treedepth for smaller coloring numbers~\cite{kun2018lincolor}.
These approaches are also starting to gain traction in interdisciplinary collaborations; a very recent
preprint shows that algorithm engineering allows the constant-factor approximation algorithm
for Distance-$r$ Dominating Set given by Dvo\v{r}\'ak to be applied to large assembly graphs in metagenomics~\cite{brown2018metagenome}.

It remains a challenge to transform efficient algorithms into practical ones (by reducing hidden constants and other trickery),
engineer scalable implementations, and forge collaborations with domain experts to ensure usability and relevance.

\paragraph*{Expected outcomes.}
The main goal of the proposed seminar is to create a platform for the exchange of tools, ideas, and questions between researchers working on different aspects of the theory.
So far, such synergy led to major developments due to the multi-disciplinary character of the field, and we hope that the seminar will foster new cross-field collaborations and lead to new results.
Besides this, the studied concepts of sparsity are very young and the new techniques are still not widely known in the related fields. 
This particularly applies to the field of algorithm design, where the toolbox seems to be applicable within multiple paradigms, whose respective communities are not aware of the new developments.
It would be desired if the seminar contributed to the visibility of the theory of sparse graph classes within neighboring areas and inspired new results based on its techniques.
Finally, the recent attempts at deploying theoretical sparsity-based methods in practice show promise and we believe that the community should strongly support applications-driven research.
We hope that the seminar will encourage more practical works such as implementation, evaluation, and fine-tuning of theoretical methods, and the usage of sparsity-based subroutines in larger systems.

\paragraph*{Structure of the seminar.}
We would like to keep the seminar oriented on research and collaboration, and therefore leave the participants plenty of time for working together.
A block of talks will be scheduled each morning, while afternoons will be kept free for collaboration.
The scientific program will be a mix of invited tutorials on important techniques, keynote lectures on recent developments, and shorter contributed talks on topics suggested by the organizers.
On the afternoon of the first day we plan an open problems session; we also would like to contact several key participants beforehand and ask them to prepare some concrete open problems.
If needed, a second open problem session will be held later during the week.

\paragraph*{Other seminars, workshops, and projects.}
There have been several events addressing different aspects of sparsity in combinatorics and algorithm design.
This demonstrates a high interest of various communities in mathematics and computer science
to foster further multidisciplnary interaction on topics covered by this proposal and
we believe that the environment of Schloss Dagstuhl is ideal to achieve this goal.
\begin{itemize}
\item A one-day satellite workshop on algorithms and structure of sparse graphs was organized in connection with the 44th International Colloquium on Automata, Languages, and Programming, ICALP 2017, held in Warsaw in July 2017.
\item A workshop on structural sparsity, logic, and algorithms was organized by Anuj Dawar, Zden\v ek Dvo\v{r}\'ak, and Daniel Kr\'a\v{l} at the University of Warwick in June 2018. This was a follow-up event of a workshop on similar topics organized at the same place in December 2016 by Daniel Kr\'a\v{l} and Ranko Lazi\'c. See \url{https://warwick.ac.uk/fac/sci/maths/people/staff/daniel_kral/strlogalg/}
and \url{https://warwick.ac.uk/fac/sci/maths/people/staff/daniel_kral/alglogstr}.
\item A one-month research project ``Research in Paris'' on {\em Structural Sparisty} was hosted by Institut Henri Poincar\'e (Paris) during May 2018, which gathered J. Ne\v set\v ril, P. Ossona de Mendez, F. Reidl, S.~Siebertz, and B. Sullivan 
\url{http://www.ihp.fr/en/activities/rip/forthcoming}.
\item In the autumn of 2018, Charles University in Prague is organizing a six-week program (DocCourse) on structural sparsity in connection with model theory offering tutorial lectures to senior undergraduate and graduate students. See \url{https://iuuk.mff.cuni.cz/events/doccourse2018/}.
\end{itemize}
We are not aware of any previous Dagstuhl seminars directly overlapping with the subject of our proposal.


\begin{footnotesize}
\bibliographystyle{abbrv}
\bibliography{references}
\end{footnotesize}

\pagebreak

\section{Invitee list}

\invitee{Isolde Adler}{University of Leeds}{UK}{I.M.Adler@leeds.ac.uk}{https://engineering.leeds.ac.uk/staff/810/Dr_Isolde_Adler}{Algorithms, logic}{junior}

\invitee{Marthe Bonamy}{LaBRI Bordeaux and CNRS}{France}{marthe.bonamy@u-bordeaux.fr}{http://www.labri.fr/perso/mbonamy/}{Combinatorics, algorithms}{female junior}

\invitee{Nicolas Bousquet}{University Grenoble Alpes and CNRS}{France}{nicolas.bousquet@grenoble-inp.fr}{https://pagesperso.g-scop.grenoble-inp.fr/~bousquen/}{Combinatorics, algorithms}{female}

\invitee{Anuj Dawar}{University of Cambridge}{UK}{anuj.dawar@cl.cam.ac.uk}{https://www.cl.cam.ac.uk/~ad260/}{Combinatorics, algorithms, logic}{}

\invitee{Reinhard Diestel}{University of Hamburg}{Germany}{ReinhardDiestel@math.uni-hamburg.de}{https://www.math.uni-hamburg.de/home/diestel/}{Combinatorics, algorithms}{}

\invitee{Zdeněk Dvořák}{Charles University, Prague}{Czechia}{ook@ucw.cz}{https://iuuk.mff.cuni.cz/~rakdver/}{Combinatorics, algorithms, logic}{}

\invitee{Eduard Eiben}{University of Bergen}{Norway}{Eduard.Eiben@uib.no}{https://www.uib.no/en/persons/Eduard.Eiben}{Algorithms}{female}

\invitee{Kord Eickmeyer}{Technical University, Darmstadt}{Germany}{eickmeyer@mathematik.tu-darmstadt.de}{https://www2.mathematik.tu-darmstadt.de/~eickmeyer/}{Algorithms, logic}{}

\invitee{Fedor Fomin}{University of Bergen}{Norway}{fomin@ii.uib.no }{http://www.ii.uib.no/~fomin/}{Algorithms}{}

\invitee{Jakub Gajarský}{Technical University, Berlin}{Germany}{jakub.gajarsky@tu-berlin.de}{http://logic.las.tu-berlin.de/Members/Gajarsky/}{Algorithms, logic}{female}

\invitee{Robert Ganian}{Technical University, Vienna}{Austria}{rganian@ac.tuwien.ac.at}{https://www.ac.tuwien.ac.at/people/rganian/}{Algorithms, logic}{female}

\invitee{Archontia Giannopoulou}{Technical University, Berlin}{Germany}{Archontia.Giannopoulou@gmail.com}{http://users.uoa.gr/~arcgian/index.html}{Combinatorics, algorithms}{female junior}

\invitee{Martin Grohe}{RWTH Aachen}{Germany}{grohe@informatik.rwth-aachen.de }{}{Combinatorics, algorithms, logic}{}

\invitee{Sariel Har-Peled}{University of Illinois in Urbana-Champaign}{USA}{sariel@uiuc.edu}{https://sarielhp.org/}{Algorithms}{}

\invitee{Petr Hliněný}{Masaryk University, Brno}{Czechia}{hlineny@fi.muni.cz}{https://www.fi.muni.cz/~hlineny/}{Algorithms, logic}{}

\invitee{Gwenaël Joret}{Free University of Brussels}{Belgium}{gjoret@ulb.ac.be}{http://di.ulb.ac.be/algo/gjoret/}{Combinatorics}{}

\invitee{Ken-ichi Kawarabayashi}{National Institute of Informatics, Tokyo}{Japan}{kkeniti@nii.ac.jp}{}{Combinatorics, algorithms}{}

\invitee{Tereza Klimošová}{Charles University, Prague}{Czechia}{tereza@kam.mff.cuni.cz}{https://iuuk.mff.cuni.cz/~tereza/}{Combinatorics}{female junior}

\invitee{Daniel Král'}{Masaryk University, Brno}{Czechia}{dkral@fi.muni.cz}{http://www.ucw.cz/~kral/}{Combinatorics, algorithms, logic}{}

\invitee{Stephan Kreutzer}{Technical University, Berlin}{Germany}{stephan.kreutzer@tu-berlin.de}{http://logic.las.tu-berlin.de/Members/Kreutzer/}{Algorithms, logic}{}

\invitee{O-joung Kwon}{Incheon National University}{South Korea}{ojoungkwon@inu.ac.kr}{http://ojkwon.com/}{Algorithms}{female}

\invitee{Aurelie Lagoutte}{University Clermont-Auvergne}{France}{aurelie.lagoutte@uca.fr}{https://fc.isima.fr/~alagoutte/}{Combinatorics}{female junior}

\invitee{Daniel Lokshtanov}{University of California in Santa Barbara}{USA}{daniello@ucsb.edu}{https://www.cs.ucsb.edu/people/faculty/lokshtanov}{Algorithms}{}

\invitee{Dániel Marx}{Hungarian Academy of Sciences (MTA SZTAKI)}{Hungary}{dmarx@cs.bme.hu }{http://www.cs.bme.hu/~dmarx/}{Combinatorics, algorithms}{}

\invitee{Claire Mathieu}{CNRS}{France}{cmathieu@di.ens.fr}{https://www.di.ens.fr/ClaireMathieu.html}{Algorithms}{}

\invitee{Piotr Micek}{Jagiellonian University, Cracow}{Poland}{piotr.micek@tcs.uj.edu.pl}{http://www.tcs.uj.edu.pl/micek}{Combinatorics}{}

\invitee{Amer Mouawad}{University of Bergen}{Norway}{amer.mouawad@gmail.com }{https://folk.uib.no/amo110/}{Algorithms}{female}

\invitee{Irene Muzi}{University of Warsaw}{Poland}{irene.muzi@gmail.com}{https://sites.google.com/site/irenemuzigraph/home}{Combinatorics, algorithms}{female junior}

\invitee{Wojciech Nadara}{University of Warsaw}{Poland}{wojtek.nadara@gmail.com }{}{Algorithms, applications}{female}

\invitee{Jaroslav Nešetřil}{Charles University, Prague}{Czechia}{nesetril@iuuk.mff.cuni.cz}{https://iuuk.mff.cuni.cz/~nesetril/en/}{Combinatorics, algorithms, logic}{}

\invitee{Jan Obdržálek}{Masaryk University, Brno}{Czechia}{obdrzalek@fi.muni.cz}{https://www.fi.muni.cz/~xobdrzal/}{Algorithms, logic}{}

\invitee{Patrice Ossona de Mendez}{CAMS Paris, CNRS, and Charles University, Prague}{France/Czechia}{patrice.ossona-de-mendez@ehess.fr }{http://cams.ehess.fr/patrice-ossona-de-mendez/}{Combinatorics, algorithms, logic}{}

\invitee{Sang-il Oum}{KAIST, Daejeon}{South Korea}{sangil@kaist.edu}{https://mathsci.kaist.ac.kr/~sangil/}{Combinatorics, algorithms}{}

\invitee{Fahad Panolan}{University of Bergen}{Norway}{Fahad.Panolan@uib.no}{https://folk.uib.no/fpa082/}{Algorithms}{female}

\invitee{Marcin Pilipczuk}{University of Warsaw}{Poland}{marcin.pilipczuk@mimuw.edu.pl}{https://www.mimuw.edu.pl/~malcin/}{Algorithms, applications}{}

\invitee{Michał Pilipczuk}{University of Warsaw}{Poland}{michal.pilipczuk@mimuw.edu.pl}{https://www.mimuw.edu.pl/~mp248287/}{Algorithms, logic}{}

\invitee{Kent Quanrud}{University of Illinois in Urbana-Champaign}{USA}{quanrud2@illinois.edu}{http://quanrud2.web.engr.illinois.edu/}{Algorithms}{female}

\invitee{Daniel Quiroz}{University of Chile, Santiago de Chile}{Chile}{dquiroz@cmm.uchile.cl}{http://www.cmm.uchile.cl/~dquiroz/}{Combinatorics}{female}

\invitee{Roman Rabinovich}{Technical University, Berlin}{Germany}{roman.rabinovich@tu-berlin.de}{http://logic.las.tu-berlin.de/Members/Rabinovich/}{Algorithms, applications}{female}

\invitee{Jean-Florent Raymond}{Technical University, Berlin}{Germany}{raymond@tu-berlin.de}{http://www.user.tu-berlin.de/jraymond/}{Combinatorics, algorithms}{female}

\invitee{Felix Reidl}{Birbeck University of London}{UK}{f.reidl@dcs.bbk.ac.uk}{https://tcs.rwth-aachen.de/~reidl/}{Combinatorics, algorithms}{female}

\invitee{Saket Saurabh}{Institute of Mathematical Sciences, Chennai}{India}{saket@imsc.res.in}{http://www.ii.uib.no/~saket/}{Algorithms}{}

\invitee{Nicole Schweikardt}{Humboldt University, Berlin}{Germany}{schweikn@informatik.hu-berlin.de}{https://www2.informatik.hu-berlin.de/~schweikn/}{Algorithms, logic}{junior}

\invitee{Luc Segoufin}{INRIA and ENS Ulm}{France}{luc.segoufin@inria.fr}{https://who.rocq.inria.fr/Luc.Segoufin/}{Algorithms, logic}{}

\invitee{Sebastian Siebertz}{Humboldt University, Berlin}{Germany}{siebertz@informatik.hu-berlin.de}{https://www.informatik.hu-berlin.de/de/institut/mitarbeiter/1691486}{Algorithms, applications}{female}

\invitee{Konstantinos Stavropoulos}{University of Hamburg}{Germany}{konstantinos.stavropoulos@uni-hamburg.de}{https://www.math.uni-hamburg.de/home/stavropoulos/}{Combinatorics}{female}

\invitee{Blair Sullivan}{North Carolina State University, Raleigh}{USA}{blair_sullivan@ncsu.edu}{https://people.engr.ncsu.edu/vbsulliv/}{Algorithms, applications}{junior}

\invitee{Dimitrios Thilikos}{LIRMM Montpellier and CNRS}{France}{sedthilk@thilikos.info}{https://www.lirmm.fr/~thilikosto/}{Combinatorics, algorithms}{}

\invitee{Szymon Toruńczyk}{University of Warsaw}{Poland}{szymtor@mimuw.edu.pl}{https://www.mimuw.edu.pl/~szymtor/}{Algorithms, logic}{}

\invitee{Torsten Ueckerdt}{Karlsruhe Institute of Technology}{Germany}{torsten.ueckerdt@kit.edu}{https://i11www.iti.kit.edu/en/members/torsten_ueckerdt/index}{Combinatorics}{female}

\invitee{Jan van den Heuvel}{London School of Economics and Political Science}{UK}{j.van-den-heuvel@lse.ac.uk}{http://www.maths.lse.ac.uk/personal/jan/}{Combinatorics}{}

\invitee{Alexandre Vigny}{University of Warsaw}{Poland}{alexandre.vigny@imj-prg.fr}{https://webusers.imj-prg.fr/~alexandre.vigny/}{Algorithms, logic}{female}

\invitee{Kristina Vušković}{University of Leeds}{UK}{K.Vuskovic@leeds.ac.uk}{https://engineering.leeds.ac.uk/staff/249/kristina_vuskovic}{Combinatorics}{junior}

\invitee{Bartosz Walczak}{Jagiellonian University, Cracow}{Poland}{walczak@tcs.uj.edu.pl}{http://www.tcs.uj.edu.pl/walczak}{Combinatorics}{}

\invitee{Daniel Weissauer}{University of Hamburg}{Germany}{TODO@TODO}{}{Combinatorics}{female}

\invitee{Paul Wollan}{University of Rome “La Sapienza”}{Italy}{wollan@di.uniroma1.it}{http://wwwusers.di.uniroma1.it/~wollan/}{Combinatorics, algorithms}{}

\invitee{David Wood}{Monash University, Melbourne}{Australia}{david.wood@monash.edu}{http://users.monash.edu.au/~davidwo/}{Combinatorics, algorithms}{}

\invitee{Marcin Wrochna}{University of Oxford}{UK}{m.wrochna@mimuw.edu.pl }{https://www.mimuw.edu.pl/~mw290715/}{Combinatorics, algorithms}{female}

\invitee{Meirav Zehavi}{Ben-Gurion University}{Israel}{zehavimeirav@gmail.com}{https://sites.google.com/site/zehavimeirav/}{Algorithms}{female junior}

\invitee{Xuding Zhu}{National Sun Yat-sen University, Kaohsiung}{Taiwan}{zhu@math.nsysu.edu.tw}{http://www.math.nsysu.edu.tw/~zhu/}{Combinatorics}{}



\pagebreak

\section{Organizers' CVs}

\heading{Personal Information}

\begin{small}
\noindent
\begin{tabular}{@{\hspace{0cm}}l @{\hspace{10mm}} p{13cm}}
{\bf Name and surname} & Daniel Kr\'al'\\[0.1cm]
{\bf Address} & Faculty of Informatics, Masaryk University, Botanick\'a 68A, 60200 Brno, Czech Republic\\
{\bf E-mail} & \verb+dkral@fi.muni.cz+\\[0.1cm]
{\bf Website} & \verb+https://www.fi.muni.cz/~dkral/+\\[0.1cm]
{\bf Research interests} & Structural and extremal graph theory, analytic methods in combinatorics,\\
                         & logic methods in graph and matroid theory and their algorithmic applications
\end{tabular}
\end{small}

\heading{Education and academic career}
\begin{small}
\noindent
\begin{tabular}{@{}p{3cm} @{\hspace{2mm}} p{14.7cm}}
Oct 2018-now & Donald Ervin Knuth Professor, Faculty of Informatics, Masaryk University, Brno\\
Sep 2018 & Professor, Faculty of Informatics, Masaryk University, Brno\\[0.1cm]
Oct 2012-now & Professor, University of Warwick (since Sep 2018 part-time)\\
& Mathematics Institute and Department of Computer Science\\
& Member of the DIMAP Centre\\[0.1cm]
Jul 2010-Sep 2012 & Associate professor, Charles University, Prague\\[0.1cm]
Feb 2011-Dec 2012 & Adjunct researcher, University of West Bohemia, Pilsen\\[0.1cm]
Aug 2006-Jun 2010 & Researcher, Charles University, Prague\\[0.1cm]
Oct 2005-Jul 2006 & Visiting assistant professor, Georgia Institute of Technology\\[0.1cm]
Aug 2005-Sep 2005 & Researcher, Charles University, Prague\\[0.1cm]
Oct 2004-Jul 2005 & Postdoctoral fellow, Technical University Berlin\\[0.1cm]
Sept 2001-Aug 2004 & PhD student, Charles University, Prague 
\end{tabular}
\end{small}

\heading{Selected scientific awards and major grants }
\begin{small}
\noindent
\begin{tabular}{@{}p{3cm} @{\hspace{2mm}} p{14.7cm}}
2014 & Philip Leverhulme Prize in Mathematics and Statistics \\[0.1cm]
2011 & European Prize in Combinatorics \\[0.1cm]
2015--2020 & ERC Consolidator grant LADIST \\[0.1cm]
2010--2015 & ERC Starting grant CCOSA
\end{tabular}
\end{small}

\heading{Professional service and commissions of trust}
\begin{small}
\noindent {\bf{Editor-in-Chief}}: SIAM J. Discrete Math. (since 2017, associate editor since 2012)
\vskip 0.1cm
\noindent {\bf{Managing editor}}: Advances in Combinatorics (since 2018)
\vskip 0.1cm
\noindent {\bf{Associate editor}}: Discrete Math. (since 2010) and Discrete Optim. (2010--2016)
\vskip 0.1cm
\noindent {\bf{Editorial board member}}: J.~Graph Theory (since 2008) and European J.~Combin. (since 2009)
\vskip 0.1cm
\noindent Member of the {\bf{steering committee of SODA}} (since 2016)
\vskip 0.1cm
\noindent {\bf{Program committee chair}}: CanaDAM'19
\vskip 0.1cm
\noindent {\bf{Program committee member}}: WG'06, MFCS'10, SODA'12, SOFSEM'12, IPEC'12, ICALP'13, EuroComb'13, SOFSEM'14, WG'14, SODA'15, EuroComb'15, FCT'15, MEMICS'15, CanaDAM'17, STACS'18 and DMD'18
\vskip 0.1cm
\noindent {\bf{Organizing committee member}}: SIAM DM'12, BCC'15, SIAM DM'16
\vskip 0.1cm
\noindent {\bf{Organization of workshops}}: CCOSA Fall School 2011, CCOSA Winter School 2013, ICMS workshop on Extremal Combinatorics 2014, LMS-CMI Research school on Regularity and Analytic Methods in Combinatorics 2015, Oberwolfach workshop on Graph Theory 2016, Workshop on Algorithms, Logic and Structure 2016, 10 Year Anniversary DIMAP Workshop 2017, Workshop in Honour of Mike Paterson's 75th Birthday 2017, Workshop on Structural Sparsity, Logic and Algorithms 2018, Oberwolfach workshop on Graph Theory 2019
\vskip 0.1cm
\noindent Member of the 2016 D\'enes K\"onig Prize selection committee
\vskip 0.1cm
\noindent Promotion or appointment reviewer for universities in Canada, Czech Republic, Germany, India, South Korea, UK and US
\end{small}

\newpage

\heading{Selected invited plenary talks}
\begin{small}
\noindent
\begin{tabular}{@{}p{1.5cm} @{\hspace{2mm}} p{16cm}}
2018 & 10th International Colloquium on Graph theory and combinatorics (ICGT), Lyon, France \\[0.1cm]
2018 & SIAM Conference on Discrete Mathematics, Denver, US \\[0.1cm]
2017 & 18th International Conference on Random Structures and Algorithms (RSA), Gniezno, Poland \\[0.1cm]
2017 & Structure in Graphs and Matroids (SIGMA), Waterloo, Canada\\[0.1cm]
2017 & Shanks Workshop: Cumberland Conference on Combinatorics, Graph Theory and Computing, Nashville, US\\[0.1cm]
%2017 & Algebraic, Topological and Complexity Aspects of Graph Covers (ATCAG), Durham, United Kingdom\\[0.1cm]
2016 & Bordeaux Graph Workshop (BGW), Bordeaux, France\\[0.1cm]
2016 & Discrete Mathematics Days, Barcelona, Spain\\[0.1cm]
%2015 & International Workshop on Graph Decompositions, Marseille, France\\[0.1cm]
2014 & International Workshop on Structure in Graphs and Matroids, Princeton, US\\[0.1cm]
2013 & Bertinoro Workshop on Algorithms and Graphs, Bertinoro, Italy\\[0.1cm]
%2013 & 22nd Workshop '3in1', Kroczyce, Poland\\[0.1cm]
%2013 & Utrecht Graphs Workshop, Utrecht, Netherlands\\[0.1cm]
2012 & Graph Theory at Georgia Tech (GTAGT), Atlanta, US\\[0.1cm]
2011 & Kolloquium \"uber Kombinatorik (Kolkom), Magdeburg, Germany\\[0.1cm]
%2011 & Joint Mathematical Conference CSASC, Krems, Austria\\[0.1cm]
2011 & 6th European Conference on Combinatorics, Graph Theory and Applications (Eurocomb), Budapest, Hungary\\[0.1cm]
2009 & 18th workshop Cycles and Colourings, Tatransk\'a \v Strba, Slovakia\\[0.1cm]
2009 & 35th Workshop on Graph-Theoretic Concepts in Computer Science (WG), Montpellier, France\\[0.1cm]
%2008 & 43th International Czech and Slovak Conference Graphs, Zadov, Czech Republic
\end{tabular}
\end{small}

\heading{Selected colloquium talks}
\begin{small}
\noindent
\begin{tabular}{@{}p{1.5cm} @{\hspace{2mm}} p{16cm}}
2018 & Department of Mathematics, Uppsala University, Sweden\\[0.1cm]
2015 & Mathematics Colloquium, University of Birmingham, United Kingdom\\[0.1cm]
2015 & School of Mathematics and ACO, Georgia Institute of Technology, Atlanta, US\\[0.1cm]
2015 & Department of Mathematics, Iowa State University, Ames, US\\[0.1cm]
2013 & Warwick Mathematics Institute, University of Warwick, United Kingdom\\[0.1cm]
2012 & Department of Mathematical Sciences Colloquium, KAIST, Daejeon, Korea\\[0.1cm]
2012 & Mathematics Colloquium, University of Illinois at Urbana-Champaign, US\\[0.1cm]
2011 & Mathematisches Kolloquium, Universit\"at Rostock, Germany\\[0.1cm]
2010 & Mathematics Colloquium, University of Ljubljana, Slovenia
\end{tabular}
\end{small}


\heading{Supervision of graduate students and postdoctoral researchers}
\begin{small}
\noindent {\bf{Supervised postdocs}}: Jean-S\' ebastien Sereni (2006--08), Louis Esperet (2008--09), Demetres Christofides (2010--11), J\'an Maz\'ak (2011--12), Andrew Treglown (2011--12), Roman Glebov (2013), Anita Liebenau (2013--15), Ping Hu (2014--2017), Tam\'as Hubai (2015--2017), P\'eter P\'al Pach (2017--18), Jonathan Noel (2017--18), Andrzej Grzesik (2017--now)
\vskip 0.1cm
\noindent {\bf{Former PhD students}}: Pavel Nejedl\'y (Charles University, 2008), Jan Hladk\'y (Charles University, 2013), Jan Volec (Warwick and Universit\'e Paris Diderot, 2014, co-advised with J.~S.~Sereni, {\em 2015 Faculty of Science Doctoral Thesis Award in Mathematics\/}), Luk\'a\v s Mach (Warwick, 2015), Tereza Klimo\v sov\'a (Warwick, 2015), Ta\'\i{}sa Lopes Martins (Warwick, 2018)
\vskip 0.1cm
\noindent {\bf{Current PhD students}}: Timothy Chan, Yanitsa Pehova
\end{small}

\heading{Selected journal publications}

\setcounter{pcount}{1}

\begin{footnotesize}
\paper{a}{J. W. Cooper, D. Kr\'al', T. Martins}{Finitely forcible graph limits are universal}{Advances in Mathematics {\bf 340} (2018), 819--854}
\paper{b}{Z. Dvo\v{r}\'ak, D. Kr\'al', R. Thomas}{Deciding first-order properties for sparse graphs}{Journal of ACM {\bf 60} (2013), article no.~36}
\paper{c}{L. Esperet, F. Kardo\v s, A. King, D. Kr\'al', S. Norine}{Exponentially many perfect matchings in cubic graphs}{ Advances in Mathematics {\bf 227} (2011), 1646--1664}
\paper{d}{D. Kr\'al', O. Pikhurko}{Quasirandom permutations are characterized by 4-point densities}{Geometric and Functional Analysis {\bf 23} (2013), 570--579}
\paper{e}{D. Kr\'al', O. Serra, L. Vena}{A Removal Lemma for systems of linear equations over finite fields}{Israel Journal of Mathematics {\bf 187} (2012), 193--207}
\end{footnotesize}

\pagebreak
\pagebreak

\heading{Personal Information}

\begin{small}
\noindent
\begin{tabular}{@{\hspace{0cm}}l @{\hspace{10mm}} p{13cm}}
{\bf Name and surname} & Michał Pilipczuk\\[0.1cm]
{\bf Date and place of birth} & June 25$^{\textrm{th}}$, $1988$, Warsaw, Poland\\[0.1cm]
%{\bf Citizenship} & Polish\\[0.1cm]
{\bf Address} & Institute of Informatics,\\
& Faculty of Mathematics, Informatics, and Mechanics of the University of Warsaw,\\
& ul. Banacha 2, 02-097 Warsaw, Poland\\[0.1cm]
%{\bf Phone number} & +48 22 5544458\\[0.1cm]
{\bf E-mail} & \verb+michal.pilipczuk@mimuw.edu.pl+\\[0.1cm]
{\bf Website} & \verb+http://www.mimuw.edu.pl/~mp248287+\\[0.1cm]
{\bf Research interests} & parameterized complexity, moderately exponential-time algorithms, kernelization,\\ &
logic in computer science, structural graph theory
\end{tabular}
\end{small}

\heading{Education and academic career}
\begin{small}
\noindent
\begin{tabular}{@{}p{3cm} @{\hspace{2mm}} p{13.2cm}}
since October 2015 & Assistant professor ({\em{pol.}} adiunkt) at the Institute of Informatics at the Faculty of Mathematics, Informatics, and Mechanics of the University of Warsaw, Poland. \\[0.2cm]
2014 --- 2015 & Postdoc of Warsaw Centre of Mathematics and Computer Science, affiliated with the Institute of Informatics, Faculty of Mathematics, Informatics, and Mechanics of the University of Warsaw,~Poland. \\[0.2cm]
2011 --- 2014 & {\em{Stipendiat}} (doctoral research fellow, a PhD position) at the Institute of Informatics of the University of Bergen, Norway, working under the supervision of prof. Fedor Fomin.
PhD thesis titled \emph{Tournaments and Optimality: New Results in Parameterized Complexity} defended in November 2013.\\[0.2cm]
%2011 --- 2013 & PhD Studies in Theoretical Computer Science under supervision of prof. Fedor V. Fomin, University of Bergen, Norway. \\[0.2cm]
2006 --- 2013 & Double Degree Program in Computer Science and Mathematics, Faculty of Mathematics, Informatics and Mechanics, University of Warsaw, Poland. 
Master thesis in computer science defended with honours in June 2011, master thesis in mathematics defended in April 2013.
%2003 --- 2006 & XIV Secondary School in Warsaw, mathematical-experimental program
\end{tabular}
\end{small}

\heading{Leadership and participation in grants}
\begin{small}
\noindent
\begin{tabular}{@{}p{3cm} @{\hspace{2mm}} p{13.2cm}}
since October 2017 & Researcher in ERC grant TOTAL ``Technology transfer between modern algorithmic paradigms'', led by Marek Cygan.\\[0.2cm]
2016 --- 2018 & Research partner in POLONEZ grant ``Algorithmic Structure Theory for Sparse Graphs'', led by Sebastian Siebertz
and funded by the National Science Center of Poland from the Horizon 2020 programme funds, Marie Sk\l{}odowska-Curie actions.\\[0.2cm]
2014 --- 2017 & Principal investigator of SONATA grant ``Optimality in Parameterized Complexity'', funded by the National Science Center of Poland.
\end{tabular}
\end{small}


\heading{Selected scientific awards}
\begin{small}
\noindent
\begin{tabular}{@{}p{1.8cm} @{\hspace{2mm}} p{14.2cm}}
2016 & ERCIM Cor Baayen Award 2016. Prize awarded annually to one promising young researcher in computer science and applied mathematics.\\[0.1cm]
2015, 2016 & START stipend granted by the Foundation for Polish Science (FNP), awarded with a~distinguishment for the highest ranked applications.\\[0.1cm]
2015 & Witold Lipski Prize for the best young researchers working in computer science in Poland.\\[0.1cm]
2015 & Stipend of the Ministry of Science and Higher Education of the Republic of Poland for outstanding young~researchers.\\[0.1cm]
2014 & Meltzer Prize for Young Researchers (Meltzerprisen for yngre forskere), awarded for the achievements in 2013.
\end{tabular}
\end{small}

\pagebreak

\heading{Selected invited talks}
\begin{small}
\noindent
\begin{tabular}{@{}p{1.5cm} @{\hspace{2mm}} p{14.7cm}}
Jun 2018 & Invited talk {\em{Parameterized algorithms for planar packing and covering problems using Voronoi diagrams}} at the workshop {\em{Fine-grained Complexity of Hard Geometric Problems}},
a satellite event of SoCG 2018. Budapest, Hungary.\\[0.1cm]
May 2018 & Invited talk {\em{From approximate to parameterized and back again: algorithms for geometric packing and covering problems using Voronoi diagrams}}
at the Lorentz Center workshop {\em{Fixed-Parameter Computational Geometry}}. Leiden, Netherlands.\\[0.1cm]
Aug 2017 & Invited talk {\em{On definable and recognizable properties of graphs of bounded treewidth}} at the 42$^{\text{nd}}$ International Symposium on Mathematical Foundations of Computer Science (MFCS 2017). Aarhus, Denmark.\\[0.1cm]
Jun 2015 & Invited talk {\em{Kernelization algorithms on sparse graph classes}} at the $5^{\textrm{th}}$ Workshop on Kernels. Nordfjordeid, Norway.\\[0.1cm]
Dec 2014 & Invited talks {\em{Algorithmic Lower Bounds based on ETH and SETH}} and {\em{Graph Isomorphism is FPT Parameterized by Treewidth}} at workshop {\em{Exact Algorithms and Lower Bounds}}, a satellite event of FSTTCS~2014. Delhi, India.\\[0.1cm]
\end{tabular}
\end{small}

\heading{Selected organization of scientific events}
\begin{small}
\noindent
\begin{tabular}{@{}p{1.6cm} @{\hspace{2mm}} p{14.6cm}}
Dec 2017 & Lecturer at the School on Recent Advances in Parameterized Complexity. Tel Aviv, Israel.\\[0.1cm]
Sep 2017 & Lecturer at the Parameterized Complexity Summer School, a satellite event of ALGO 2017. Vienna, Austria.\\[0.1cm]
Jul 2017 & Co-organizer (with Sebastian Siebertz) of a satellite workshop of ICALP 2018 on algorithms and structure for sparse graphs. Warsaw, Poland.\\[0.1cm]
Jun 2016 & Co-organizer (with Marek Cygan and Marcin Pilipczuk) of sessions on parameterized complexity during SIAM Conference on Discrete Mathematics 2016. Atlanta,~USA.\\[0.1cm]
Aug 2014 & Organizer and lecturer at the International School on Parameterized Algorithms. Bedlewo, Poland.
\end{tabular}
\end{small}

\heading{Service}
\begin{small}
\noindent {\bf{Supervised PhD students}}: Marcin Wrochna (graduated in 2018)
\vskip 0.1cm
\noindent {\bf{Program Committee member of}}: IPEC 2018 (co-chair), ICALP 2018 (track A), STOC 2018, STACS 2018, IPEC 2017, ESA 2016 (track A), WG 2016, ICALP 2015, WALCOM 2015, FSTTCS 2014
\end{small}

\heading{Selected publications connected with the proposal's topic}

\setcounter{pcount}{1}

\begin{footnotesize}
\paperbr{wide-stab}{J. Gajarsk\'y, S. Kreutzer, J. Ne\v{s}et\v{r}il, P. Ossona de Mendez, M. Pilipczuk, S. Siebertz, Sz. Toru\'nczyk}
{First-order interpretations of bounded expansion classes}{Proceedings of the 45$^{\text{th}}$ International Colloquium on Automata, Languages, and Programming, ICALP 2018}

\paper{wide-stab}{M. Pilipczuk, S. Siebertz, Sz. Toruńczyk}
{On the number of types in sparse graphs}{Proceedings of the 23$^{\textrm{rd}}$ Annual ACM/IEEE Symposium on Logic in Computer Science, LICS 2018}

\paperbr{succ-inv}{J. van den Heuvel, S. Kreutzer, M. Pilipczuk, D. A. Quiroz, R. Rabinovich, S. Siebertz}
{Model-checking for successor-invariant first-order formulas on graph classes of bounded expansion}{Proceedings of the 22$^{\textrm{nd}}$ Annual ACM/IEEE Symposium on Logic in Computer Science, LICS 2017}

\paperbr{nei-comp}{K. Eickmeyer, A. Giannopoulou, S. Kreutzer, O. Kwon, M. Pilipczuk, R. Rabinovich, S. Siebertz}
{Neighborhood complexity and kernelization for nowhere dense classes of graphs}{Proceedings of the 44$^{\text{th}}$ International Colloquium on Automata, Languages, and Programming, ICALP 2017}

\paperbr{bounded-exp}{P. G. Drange, M. S. Dregi, F. V. Fomin, S. Kreutzer, D. Lokshtanov, M. Pilipczuk, M. Pilipczuk, F. Reidl, F. Sánchez Villaamil, S. Saurabh, S. Siebertz, S. Sikdar}
{Kernelization and Sparseness: the case of Dominating Set}{Proceedings of the 33$^{\textrm{rd}}$ International Symposium on Theoretical Aspects of Computer Science, STACS 2016}
\end{footnotesize}

\pagebreak
\heading{Personal Information}

\begin{small}
\noindent
\begin{tabular}{@{\hspace{0cm}}l @{\hspace{5mm}} p{12cm}}
{\bf Name and surname} & Sebastian Siebertz\\[0.1cm]
{\bf Date and place of birth} & February 20$^{\textrm{th}}$, $1984$, Bergisch-Gladbach, Germany\\[0.1cm]
%{\bf Citizenship} & German\\[0.1cm]
{\bf Address} & Institut f\"ur Informatik, Chair for Algorithm Engeneering, Humbold-Universit\"at zu Berlin, Johann-von-Neumann-Haus, Rudower Chausse 25, 
D-12489 Berlin-Adlershof, Germany\\[0.1cm]
%{\bf Phone number} & +49 30 3093 3010\\[0.1cm]
{\bf E-mail} & \verb+siebertz@informatik.hu-berlin.de+\\[0.1cm]
{\bf Website} & \verb+http://www.mimuw.edu.pl/~siebertz+\\[0.1cm]
{\bf Research interests} & 
Logic in computer science, structural graph theory, 
parameterized complexity theory
\end{tabular}
\end{small}

\heading{Education and academic career}
\begin{small}
\noindent
\begin{tabular}{@{}p{3cm} @{\hspace{2mm}} p{13.2cm}}
since October 2018 & Postdoc at the Chair for Algorithm Engeneering 
 at Humboldt-Universit\"at zu Berlin \\[0.2cm]
2016 --- 2018 & Marie Sk\l odowska-Curie Fellow (supported by the National Science Centre of Poland and the European Union's Horizon 2020 research and 
innovation  programme) at the Institute of Informatics of the University of
    Warsaw. \\[0.2cm]
2015 --- 2016 & Postdoc at the 
chair for Logic and Semantic at Technical University Berlin (TU Berlin), Germany.\\[0.2cm]
2011 --- 2015 & Doctoral research fellow (PhD position) at the 
chair for Logic and Semantic at
Technical University Berlin (TU Berlin), Germany, working under the supervision of Prof.\ Stephan Kreutzer.
PhD thesis titled \emph{Nowhere Dense Classes of Graphs: Characterisations and Algorithmic Meta-Theorems} defended in September 2015.  Parental leave from February 2014 --- February 2015.\\[0.2cm]
2004 --- 2011 & Studies in computer science at RWTH 
Aachen University. Diploma in computer science defended
in April 2011. 

\end{tabular}
\end{small}

\heading{Leadership and participation in grants}
\begin{small}
\noindent
\begin{tabular}{@{}p{3cm} @{\hspace{2mm}} p{13.2cm}}
2016 --- 2018 & Principal investigator in POLONEZ grant ``Algorithmic Structure Theory for Sparse Graphs'', funded by the National Science Center of Poland from the Horizon 2020 programme funds, Marie Sk\l{}odowska-Curie actions.\\[0.2cm]
\end{tabular}
\end{small}

\heading{Selected talks}
\begin{small}
\noindent
\begin{tabular}{@{}p{1.5cm} @{\hspace{2mm}} p{14.5cm}}
Jul 2018 & Talk \emph{On the number of types in sparse graphs} at
the ACM/IEEE Symposium on Logic in Computer Science (LICS 2018),
Oxford, UK.  \\[0.1cm]
Jan 2018 & Talk \emph{Polynomial Kernels and Wideness Properties of
Nowhere Dense Graph Classes} at the ACM-SIAM Symposium on Discrete Algorithms (SODA 2017), Barcelona, Spain. \\[0.1cm]
Sep 2017 & Invited talk \emph{First-Order Model-Checking} at the 
parameterized complexity summer school, Sep~{1--3}, 2017, Vienna, Austria.\\[0.1cm]
Jul 2014 & Invited talk \emph{Deciding first-order properties of
 nowhere dense graphs} at the Midsummer Combinatorial Workshop XX, July 28 -- Aug 1, 2014, Charles University, Prague, Czech Republic.
\end{tabular}
\end{small}

\heading{Organization of scientific events}
\begin{small}
\noindent
\begin{tabular}{@{}p{1.6cm} @{\hspace{2mm}} p{15.9cm}}
Jul 2017 & Co-organizer (with Michał Pilipczuk) of a satellite workshop of ICALP 2018 on algorithms and structure for sparse graphs. Warsaw, Poland.\\[0.1cm]
Sep 2015 & Local co-organizer of \emph{Computer Science Logic} (CSL 2015), Berlin, Germany.
\end{tabular}
\end{small}

\pagebreak

\heading{Service}
\begin{small}
\noindent {\bf{Co-supervised Bachelor students}}:\\ Moritz Zielke, 
Thesis titled \emph{Empirical Evaluation of Splitter-Game Based Algorithms for the Dominating Set Problem.}\\
Alexander Court, Thesis titled \emph{Empirical Evaluation of Approximation Algorithms for Graph Colouring Numbers.}\\
Frank Dehne, Thesis titled \emph{Empirical Evaluation of Approximation Algorithms for Directed Tree-Width.}
\vskip 0.1cm
\noindent {\bf{Program Committee member of}}: CSL 2018, Highlights of Logics Games and Automata 2018.
\end{small}

\heading{Selected publications connected with the proposal's topic}

\setcounter{pcount}{1}

\begin{footnotesize}

\paper{mc}{Martin Grohe, Stephan Kreutzer, Sebastian Siebertz}
{Deciding first-order properties of nowhere dense graphs}{Journal of the ACM, 2017}

\paperbr{wide-stab}{Jakub Gajarsk\'y, Stephan Kreutzer, Jaroslav Ne\v{s}et\v{r}il, Patrice Ossona de Mendez, Micha\l{} Pilipczuk, Sebastian Siebertz, Szymon Toru\'nczyk}
{First-order interpretations of bounded expansion classes}{Proceedings of the 45$^{\text{th}}$ International Colloquium on Automata, Languages, and Programming, ICALP 2018}

\paper{wide-stab}{Michał Pilipczuk, Sebastian Siebertz, Szymon Toruńczyk}
{On the number of types in sparse graphs}{Proceedings of the 23$^{\textrm{rd}}$ Annual ACM/IEEE Symposium on Logic in Computer Science, LICS 2018}


\paperbr{nei-comp}{Kord Eickmeyer, Archontia Giannopoulou, Stephan Kreutzer, O-joung Kwon, Michał Pilipczuk, Roman Rabinovich, Sebastian Siebertz}
{Neighborhood complexity and kernelization for nowhere dense classes of graphs}{Proceedings of the 44$^{\text{th}}$ International Colloquium on Automata, Languages, and Programming, ICALP 2017}

\paperbr{bounded-exp}{Pål Grønås Drange, Markus S. Dregi, Fedor V. Fomin, Stephan Kreutzer, Daniel Lokshtanov, Marcin Pilipczuk, Michał Pilipczuk, Felix Reidl, Fernando Sánchez Villaamil, Saket Saurabh, Sebastian Siebertz, Somnath Sikdar}
{Kernelization and Sparseness: the case of Dominating Set}{Proceedings of the 33$^{\textrm{rd}}$ International Symposium on Theoretical Aspects of Computer Science, STACS 2016}
\end{footnotesize}

\vfill
\vfill
\vfill
\vfill
\vfill
\vfill
\vfill
\vfill
\vfill
\vfill
\vfill
\vfill
\pagebreak
\heading{Personal Information}

\begin{small}
\noindent
\begin{tabular}{@{\hspace{0cm}}l @{\hspace{10mm}} p{13cm}}
{\bf Name and surname} & Blair D. Sullivan\\[0.1cm]
{\bf Address} & Department of Computer Science, Box 8206, North Carolina State University,\\
& Raleigh, NC 27695\\[0.1cm]
{\bf Phone number} & +1 919 513 0453\\[0.1cm]
{\bf E-mail} & \verb+blair_sullivan@ncsu.edu+\\[0.1cm]
{\bf Website} & \verb+http://www.csc.ncsu.edu/faculty/bdsullivan+\\[0.1cm]
\end{tabular}
\end{small}

\heading{Education and academic career}
\begin{small}
\noindent
\begin{tabular}{@{}p{3cm} @{\hspace{2mm}} p{13.3cm}}
since August 2016 & Associate Professor, Department of Computer Science, North Carolina State University \\[0.2cm]
2013 --- 2016 & Assistant Professor, Department of Computer Science, North Carolina State University \\[0.2cm]
2008 --- 2013 & Research \& Development Staff, Computer Science \& Mathematics Div., Oak Ridge National Laboratory \\[0.2cm]
2007 -- 2008 & Visiting Researcher, R\'enyi Institute, Budapest, Hungary\\[0.2cm]
2003 -- 2008 & Ph.D. in Mathematics, Department of Mathematics, Princeton University \\[0.2cm]
1999 --- 2003 & B.Sc. in Computer Science, B.Sc. in Applied Mathematics, Georgia Institute of Technology.
\end{tabular}
\end{small}

\heading{Selected scientific awards}
\begin{small}
\noindent
\begin{tabular}{@{}p{1.8cm} @{\hspace{2mm}} p{14.7cm}}
2013 & National Consortium for Data Science Faculty Fellow.\\[0.1cm]
2003 -- 2007 & Department of Homeland Security Graduate Fellowship \& Dissertation Award.\\[0.1cm]
2003 & Phi Kappa Phi Scholarship Cup, awarded to Georgia Tech senior with most outstanding academic record.\\[0.1cm]
2003 & University System of Georgia Outstanding Scholar.\\[0.1cm]
1999 -- 2003 & Georgia Tech Presidential Scholar and Jo Baker Scholar (2003).
\end{tabular}
\end{small}


\heading{Selected grants}
\begin{small}
\noindent
\begin{tabular}{@{}p{3cm} @{\hspace{2mm}} p{13.3cm}}
since October 2014 & Data-Driven Discovery Investigator Award, Gordon \& Betty Moore Foundation (\$1.5M USD).\\[0.2cm]
2016 --- 2018 & Principal Investigator of grant ``Algorithms for Exploiting Approximate Network Structure'', funded by Army Research Office (\$540K USD).\\[0.2cm]
2014 --- 2017 & Principal investigator of grant ``Parameterized Algorithms Respecting Structure in Noisy Graphs (PARSiNG)'', funded by DARPA (~\$250K USD).\\[0.2cm]
2009 --- 2012 & Principal investigator of grant ``Scalable Graph Decomposition and Algorithms to Support the Analysis of Petascale Data'', funded by US Department of Energy (~\$1.2M USD).\\[0.2cm]
\end{tabular}
\end{small}


\heading{Selected leadership/organization of scientific events}
\begin{small}
\noindent
\begin{tabular}{@{}p{2.6cm} @{\hspace{2mm}} p{13.7cm}}
2016 -- present & Steering Committee, SIAM Workshop on Network Science.\\
Jul 2016 & Co-Chair (with J. Gilbert) of SIAM Workshop on Network Science, Boston, Massachusetts.\\[0.1cm]
Apr 2016 & Co-Organizer (w/ C. Greene, B. King, M. Turk) of Barnraising for Data-Intensive Discovery 2016 at Maine Developmental and Integrative Biology Laboratory (MDIBL), Bar Harbor, Maine.\\[0.1cm]
2015 & Organizing Committee for SIAM Conference on Discrete Mathematics.\\[0.1cm]
Apr 2014 & Co-Organizer (w/ E. Demaine, D. Marx) ICERM Research Cluster: Towards Efficient Algorithms Exploiting Graph Structure, Providence Rhode Island.\\[0.1cm]
\end{tabular}
\end{small}

\pagebreak

\heading{Selected invited talks}
\begin{small}
\noindent
\begin{tabular}{@{}p{1.5cm} @{\hspace{2mm}} p{14.7cm}}
Dec 2018 & Keynote Speaker at the 12th International Conference on Combinatorial Optimization and Applications (COCOA). Atlanta, Georgia.\\[0.1cm]
Oct 2018 & Mathematics Department Colloquium at Georgia Institute of Technology. Atlanta, Georgia.\\[0.1cm]
Jul 2018 & Invited talk at Workshop on Structural Sparsity, Logic, and Algorithms. Warwick, United Kingdom.\\[0.1cm]
Jun 2016 & Invited talk in {\em Mathematics behind Big Data Analysis Minisymposium} at SIAM Conference on Discrete Mathematics. Atlanta, Georgia.\\[0.1cm]
Jul 2015 & Invited talk at {\em Mathematics for Data Science Workshop} at ICERM. Providence, Rhode Island.\\[0.1cm]
Feb 2015 & Program in Applied \& Computational Mathematics Colloquium at Princeton University. Princeton, New Jersey.\\[0.1cm]
Jan 2015 & Invited talk at {\em Workshop on the Mathematics of Network Science}, AMS/MAA Joint Mathematics Meetings. San Antonio, TX.\\[0.1cm]
\end{tabular}
\end{small}

\heading{Selected publications connected with the proposal's topic}

\setcounter{pcount}{1}

\begin{footnotesize}

\paperbr{wide-stab}{C. T. Brown, D. Moritz, M. P. O'Brien, F. Reidl, T. E. Reiter, B. D. Sullivan.}{Exploring neighborhoods in large metagenome assembly graphs reveals hidden sequence diversity.}{bioRxiv:10.1101/462788.}

\paper{wide-stab}{J. Kun, M. P. O'Brien, B. D. Sullivan.}{Treedepth Bounds in Linear Colorings.}{Proceedings of 44th International Workshop on Graph-Theoretic Concepts in Computer Science (WG), 2018. ArXiv:1802.09665v3.}

\paperbr{wide-stab}{K. Kloster, P. Kuinke, {M. P. O'Brien}, F. Reidl, F. Sanchez Villaamil, B. D. Sullivan, {A. van der Poel}}{A practical algorithm for Flow Decomposition and transcript assembly.}{Proceedings of Algorithm Engineering \& Experiments (ALENEX) 2018. ArXiv:1706.07851}

\paper{wide-stab}{M. P. O'Brien, B. D. Sullivan}{An experimental evaluation of a bounded expansion algorithmic pipeline.}{ArXiv:1712.06690.}

\paper{wide-stab}{Irene Muzi, {M. P. O'Brien}, F. Reidl, B. D. Sullivan}{Being even slightly shallow makes life hard.}{Proceedings of Mathematical Foundations of Computer Science (MFCS) 2017. ArXiv:1705.06796.}

\paperbr{wide-stab}{E. Demaine, F. Reidl, P. Rossmanith, F. S. Villaamil, S. Sikdar, B. D. Sullivan}{Structural Sparsity of Complex Networks: Random Graph Models and Linear Algorithms}{ArXiv:1406.2587.}

\end{footnotesize}

\heading{Selected open source software connected with the proposal's topic}

\setcounter{pcount}{1}

\begin{footnotesize}

\software{wide-stab}{C. T. Brown, D. Moritz, M. P. O'Brien, F. Reidl, B. D. Sullivan.}{SpaceGraphCats.}{\texttt{https://github.com/spacegraphcats/spacegraphcats} doi:10.5281/zenodo.1478025}{Python package for efficiently computing hierarchy of $r$-dominating graphs that summarize the neighborhood structure of a sparse graph at multiple resolutions.  Includes functionality for fast extraction of the neighborhood around a set of query vertices.\looseness-1}

\software{wide-stab}{M. P. O'Brien, C. G. Hobbs, K. Jasnick, F. Reidl, B. Mork, N. G. Rodrigues, B. D. Sullivan}{CONCUSS: Combatting Network Complexity Using Structural Sparsity.}{\texttt{https://www.github.com/theoryinpractice/concuss} doi:10.5281/zenodo.55690}{Python software package providing proof-of-concept for an end-to-end pipeline for parameterized analytics in bounded expansion classes. Current modules use low-treedepth colorings to support subgraph isomorphism counting (motif counting).}

\end{footnotesize}


\heading{Service}
\begin{small}
\noindent \textbf{Mentees.} PhD students: Michael O'Brien (graduated in 2018), Andrew van der Poel (anticipated defense in 2019), Brian Lavallee. Postdocs: Kyle Kloster (2016--2018), Felix Reidl (2016--2017).
\vskip 0.1cm
\noindent \textbf{Outreach.}  AWM Research Symposium;  NCSU Women in CS; ORNL Women in Computing Advisory Board; SUMMER@ICERM; SAMSI E\&O Workshops; Data Science seminars at RTI, NIEHS, RTP180; RTP R-Ladies Women in Data Science Panel, etc.
\vskip 0.1cm
\noindent \textbf{Program Committees.} Complex Networks 2018; ALENEX 2017; SIAM NS 2017, 2015; SIAM DM 2016; SIAM CSC 2014.
\end{small}


\section{Proposed dates}

TODO




\end{document}
