\paragraph*{Applied eyes of an algorithm designer.}

Over the last twenty years, the field of network science
has burgeoned, developing new methods for complex network data arising
in diverse fields including social networks, bioinformatics, quantum computing,
transportation, and healthcare. Surprisingly, few tools from structural graph theory have been assimilated into
their arsenal. In part, this is due to the theoretical nature of
much of the related literature on parameterized graph algorithms
and a lack of cross-pollination of the research
communities.

There has been some work on making structure-based approaches more practical, mostly focused on
bounded treewidth and planar graphs. When working with bounded treewidth, one significant
challenge is finding low-width tree decompositions -- here, work by Bodlaender and Koster (e.g.~\cite{bodlaender2006-tw, koster2001-tw})
set the stage for the PACE challenge (https://pacechallenge.org). More recently, there have also been
experimental evaluations of treewidth-based algorithms for solving downstream optimization problems like
Vertex Cover and Hamiltonian Cycle ~\cite{ziobro2018hamcycle-tw, alber2005vc-tw}. In an even more
applied setting, recently treewidth solvers were shown to be competitive for finding contraction orderings
for tensor networks used in quantum computing simulations~\cite{dumitrescu2018tensors}. There has also been some
work on parallel algorithms for bounded treewidth graphs, including a distributed memory implementation of a
Maximum Weighted Independent Set solver~\cite{sullivan2013paralleltd}.
Planar graphs have also attracted more practical interest, in part due to their natural representation of problems with geographical
constraints (e.g. ~\cite{alber2001-planar, schmidt2009-planarvision}).

Since the introduction of structural sparsity and bounded expansion,
there has been a resurgence in interest, in part because of work showing that
this class includes many real-world networks (e.g. \cite{demaine2018sparsity}).
Similar to algorithms for bounded treewidth, pipelines for structurally sparse graph classes
require computation of a ``decomposition,'' such as a low-treedepth coloring. An end-to-end implementation
of the algorithm for subgraph isomorphism counting~\cite{obrien2017concuss} revealed that the
number of colors used in the current constant-factor approximation algorithms was causing a significant
practical bottleneck. Subsequently, there has been work on comparing algorithms and heuristics
for computing these parameters~\cite{wojciech2018quasiwide}, as well as more theoretical research
on alternative colorings that trade off treedepth for smaller coloring numbers~\cite{kun2018lincolor}.
These approaches are also starting to gain traction in interdisciplinary collaborations; a very recent
preprint shows that algorithm engineering allows the constant-factor approximation algorithm
for $r$-dominating set given by Dvorak to be applied to large assembly graphs in metagenomics~\cite{brown2018metagenome}.

It remains a challenge to transform efficient algorithms into practical ones (by reducing hidden constants and other trickery),
engineer scalable implementations, and forge collaborations with domain experts to ensure usability and relevance.
