\paragraph*{Eyes of a model theorist.}
The theory of nowhere dense graphs is intimately linked to model
theory and finite model theory. The connection to finite model
theory was first established by Dawar, who introduced the notion
of \emph{uniform quasi-wideness} and proved that the homomorphism
preservation theorem holds for finite structures that are uniformly
quasi-wide~\cite{dawar2010homomorphism}. It was later
observed by Ne\v{s}et\v{r}il and Ossona de Mendez that for graphs
the notions of uniform quasi-wideness and nowhere denseness
coincide~\cite{nevsetvril2010first}. 

In the sequel, the characterization of nowhere dense classes via
uniform quasi-wideness turned out to be very useful, especially 
for the task of algorithmically testing first-order properties of
graphs. The problem of determining whether a formula $\varphi$
of some logic $\mathcal{L}$ is true on a given structure is known 
as the model-checking problem for $\mathcal{L}$. By proving
tractability of the model-checking problem for a logic $\mathcal{L}$
one establishes tractability for a large number of problems, namely
for all problems that can be formulated in $\mathcal{L}$, such 
tractability results are often called \emph{algorithmic meta 
theorems}. It was shown by Dvo\v{r}\'ak et al.~\cite{DvorakKT13}
that every first-order property of graphs can be decided in linear time
on every fixed class of graphs of bounded expansion, and by Grohe et 
al.~\cite{grohe2017deciding} that every first-order property of 
graphs can be decided in almost linear time on every fixed nowhere
dense class of graphs. Phrased in terms of parameterized
complexity theory, these results state that the first-order 
model-checking problem is fixed-parameter tractable, parameterized
by the length of the input formula, on every bounded bounded
expansion or nowhere dense class of graphs. On the other hand, 
nowhere dense classes that are closed under taking subgraphs 
form the border of tractability for the model-checking problem. 
It was shown by Dvo\v{r}\'ak et al.~\cite{DvorakKT13} and
Kreutzer~\cite{kre11} that model-checking on classes that
are closed under taking subgraphs and that are not nowhere
dense is as hard as on general graphs, in particular, it is 
assumed to be not fixed-parameter tractable. 
Hence, on the one hand, the above algorithmic meta theorems 
help to understand the essence of sparsity
based algorithmic techniques by abstracting from problem-specific
details. On the other hand, the corresponding hardness results show 
the limitations of these techniques beyond sparse graphs. 

Recent research aims to extend the model-checking result
beyond nowhere dense graph classes that are not closed 
under taking subgraphs. One approach is to study classes
of graphs that are \emph{structurally sparse} in the sense
that they are first-order interpretations of sparse graphs. 
In this direction, it has been proved that model checking is 
fixed parameter tractable on map graphs~\cite{eickmeyer2017fo}, 
and on interpretations of classes of graphs with bounded 
maximum degree~\cite{gajarsky2016new}. On a structural 
level, it was shown that interpretations of bounded expansion 
classes of graphs admit low shrub-depth decompositions, 
a dense analog of low tree-depth 
decompositions~\cite{gajarsky2018first}.

A second approach to extending the model-checking results 
is based on another deep connection of
nowhere denseness with classical model theory, more precisely
with \emph{stability theory}. It was observed by Adler and 
Adler~\cite{adler2014interpreting} that nowhere denseness
corresponds to the model-theoretic notion of 
\emph{superflatness}. Furthermore, it was observed that nowhere
dense classes are \emph{stable}, a key notion in Shelah's 
classification theory~\cite{shelah1990classification}. 
Surprisingly, on subgraph closed classes of graphs the notions
of stability and nowhere denseness coincide. This connection
has already proved its efficiency in transferring techniques and 
notions from model theory to combinatorics and algorithms, 
see for instance \cite{siebertz2016polynomial, 
malliaris2014regularity, pilipczuk2018number}. 

%Model theory  is the study of classes of mathematical structures from the perspective of mathematical logic, and this study mainly concerns infinite structures.
%A classification of theories based on those whose models can be classified and those whose models are too complicated to classify has been initiated by Shelah, who  drew two important dividing lines: NIP vs dependent and stable vs unstable \cite{shelah1990classification}.
%
%Finite model theory, which focuses on finite structures,  diverges significantly from the study of infinite structures in both the problems studied and the techniques used, and mainly grew out of computer science applications.  This led to the development of strong tools to study logics over finite structures, which helped to answer many questions about  complexity theory, databases, formal languages, algorithms, and many more. This also led to the notion of {\em tame} classes of finite structures, which behave well.  In this line Dawar \cite{Dawar2010} introduced the notion of a \emph{uniform quasi-wide class} and proved 
%     that the homomorphism preservation theorem holds when relativized to a uniformly quasi-wide class of finite structures. It was later shown 
%by Ne\v{s}et\v{r}il and Ossona de Mendez  that the uniform quasi-wide classes are exactly nowhere dense classes \cite{ND_logic}.
%
%The  structural and algorithmic properties of  nowhere dense classes  are deeply connected to  the model theory notions of independence and stability, as witnessed in particular by the following two fundamental results. The first asserts that for a monotone class of graphs, the notions of nowhere dense class, NIP class, and stable class are equivalent \cite{adler2014interpreting}. The second states that nowhere dense classes are the most general monotone classes for which first-order model checking is fixed-parameter tractable \cite{grohe2017deciding}. This connection has already proved its efficiency in transferring techniques and notions from model theory to combinatorics (see for instance \cite{pilipczuk2018number}) and, conversely, from combinatorics to finite model theory (see for instance \cite{rossman2008homomorphism}).
%
%Numerous algorithmic applications of structural properties of the graphs in a nowhere dense class (or, more restrictively in a bounded expansion class) have appeared, including   a near-optimal kernelization algorithm for the distance-$r$ dominating set problem for the graphs in a nowhere dense class \cite{eickmeyer2016neighborhood}.
%

