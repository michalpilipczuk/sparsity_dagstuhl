\paragraph*{Eyes of a model theorist.}
Model theory  is the study of classes of mathematical structures from the perspective of mathematical logic, and this study mainly concerns infinite structures.
A classification of theories based on those whose models can be classified and those whose models are too complicated to classify has been initiated by Shelah, who  drew two important dividing lines: NIP vs dependent and stable vs unstable \cite{shelah1990classification}.

Finite model theory, which focuses on finite structures,  diverges significantly from the study of infinite structures in both the problems studied and the techniques used, and mainly grew out of computer science applications.  This led to the development of strong tools to study logics over finite structures, which helped to answer many questions about  complexity theory, databases, formal languages, algorithms, and many more. This also led to the notion of {\em tame} classes of finite structures, which behave well.  In this line Dawar \cite{Dawar2010} introduced the notion of a \emph{uniform quasi-wide class} and proved 
     that the homomorphism preservation theorem holds when relativized to a uniformly quasi-wide class of finite structures. It was later shown 
by Ne\v{s}et\v{r}il and Ossona de Mendez  that the uniform quasi-wide classes are exactly nowhere dense classes \cite{ND_logic}.

The  structural and algorithmic properties of  nowhere dense classes  are deeply connected to  the model theory notions of independence and stability, as witnessed in particular by the following two fundamental results. The first asserts that for a monotone class of graphs, the notions of nowhere dense class, NIP class, and stable class are equivalent \cite{adler2014interpreting}. The second states that nowhere dense classes are the most general monotone classes for which first-order model checking is fixed-parameter tractable \cite{grohe2017deciding}. This connection has already proved its efficiency in transferring techniques and notions from model theory to combinatorics (see for instance \cite{pilipczuk2018number}) and, conversely, from combinatorics to finite model theory (see for instance \cite{rossman2008homomorphism}).

Numerous algorithmic applications of structural properties of the graphs in a nowhere dense class (or, more restrictively in a bounded expansion class) have appeared, including   a near-optimal kernelization algorithm for the distance-$r$ dominating set problem for the graphs in a nowhere dense class \cite{eickmeyer2016neighborhood}.

Guided by the model theory intuition, it is natural to expect that some of the techniques and results proved for monotone sparse classes may extend to hereditary structurally sparse classes, that is possibly dense classes of low structural complexity. In this direction, it has been proved that model checking is fixed parameter tractable on map graphs \cite{eickmeyer2017fo} and on interpretations of classes of graphs with bounded maximum degree \cite{gajarsky2016new} and that transductions of bounded expansion classes of graphs admit low shrub-depth decompositions, a dense analog of low tree-depth decompositions \cite{gajarsky2018first}. 

