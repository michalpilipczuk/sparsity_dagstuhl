\paragraph*{Model theory.}
The  structural and algorithmic properties of sparse classes of graphs --- notably nowhere dense classes --- are deeply connected to  the model-theoretical notions of independence and stability, as witnessed in particular by two fundamental results. The first result asserts that for a monotone class of graphs, the notions of nowhere dense class, NIP class, and stable class are equivalent \cite{adler2014interpreting}. The second one states that nowhere dense classes are the most general monotone classes for which first-order model checking is fixed-parameter tractable \cite{grohe2017deciding}. This connection has already proved his efficiency in transferring techniques and notions from model theory to combinatorics (see for instance \cite{pilipczuk2018number}) and, conversely, from combinatorics to finite model theory (see for instance \cite{rossman2008homomorphism}).

Guided by the model theoretical intuition, it is natural to expect that some of the techniques and results proved for monotone sparse classes may extend to hereditary structurally sparse classes, that is possibly dense classes of low structural complexity. In this direction, it has been proved that model checking is fixed parameter tractable on interpretations of classes of graphs with bounded maximum degree \cite{gajarsky2016new} and that transductions of bounded expansion classes of graphs admit low shrub-depth decompositions, a dense analog of low tree-depth decompositions \cite{gajarsky2018first}. 

The notion of nowhere denseness is intimately linked to 
model theory and finite model theory. 
Model theory is the study of logics on classes of 
mathematical structures, and finite model theory restricts this
study to finite structures. Nevertheless, finite model theory is
a separate subject, and not just a small chapter in classical
model theory. One of the reasons is that many standard 
methods of classical model theory fail when only finite structures 
are considered. These include the compactness theorem, the 
completeness theorem and various interpolation and preservation
theorems. Another reason is that finite model theory is an area 
that grew out of computer science applications. This led to
the development of strong tools to study logics over finite 
structures, which helped to answer many questions about 
complexity theory, databases, formal languages, algorithmics,
and many more. 

The failure of the classical tools from model theory on the class 
of all finite structures led researchers to investigate whether
there are subclasses of finite structures that may be 
better behaved. In finite model 
theory such classes are called \emph{tame}. One success story 
in this line of research is the study of homomorphism
preservation theorems in the finite. A result of classical model 
theory states that a first-order formula is preserved under
homomorphisms on all structures if, and only if, it is logically 
equivalent to an existential positive formula. A surprising 
result by Rossmann (LICS 2005) shows that this result also holds when 
restricted to finite structures. Furthermore, Dawar (JCSS 2010) introduced
the notion of \emph{uniform quasi-wideness} and proved 
that the homomorphism preservation theorem holds on all
uniformly quasi-wide finite structures. It was later observed 
by Ne\v{s}et\v{r}il and Ossona de Mendez (JSL 2010) that this notion 
of model-theoretic tameness corresponds exactly to the 
graph-theoretic notion of nowhere denseness. A result that
sparked much interest of finite model theorists for nowhere
dense graph classes.

The model-theoretic notion of uniform quasi-wideness was 
also soon picked up in algorithmic research. Using a constructive, 
algorithmic version of uniform quasi-wideness, it was shown
by Dawar and Kreutzer (FSTTCS 2009) that various algorithmic domination and 
and independence problems can be solved efficiently on 
nowhere dense graph classes. 

Besides providing tools for solving
individual problems efficiently on restricted graph classes, 
model theory offers a very elegant 
approach to explain algorithmic techniques that work not only for 
individual problems but for whole classes of problems. This
approach is to classify problems by their descriptive complexity, 
that is, by the resources required to describe the problem in a 
suitable logic. Because tractability results for logically defined 
classes of problems establish tractability for a large number of
problems, such results are often called \emph{algorithmic meta theorems}. The prototypical example of an algorithmic
meta theorem is Courcelle’s Theorem, stating that 
every property of graphs that is definable in monadic second-order 
logic is decidable in linear time on every class of bounded tree-width.
There is a long line of meta theorems for first-order logic on sparse
structures, culminating in the result of Grohe et al.\ (STOC 2014), which states
that every property of graphs that is definable in first-order logic 
is decidable in nearly linear time on every nowhere dense class 
of graphs. This result (under standard complexity theoretic
assumptions) cannot be extended to
classes which are not nowhere dense and closed under taking
subgraphs, as shown by Kreutzer (ECCC 2009) and Dvo\v{r}\'ak et al.\ (JACM 2013)

Hence, under
standard complexity theoretic assumptions, the classification of
subgraph closed graph classes that admit efficient first-order 
model-checking is complete. Consequently, recent research 
has shifted to investigating the complexity of the model-checking
problem for first-order logic on dense classes which are not closed 
under taking subgraphs. And again, it is model theory which may
guide the way for this line of research. Adler and Adler (EJC 2014) observed
that the notion of nowhere denseness is strongly connected 
to the notion of \emph{stability}, which is a key dividing line 
in classical model theory between well-behaved and complicated 
first-order theories. More precisely, they showed that 
every nowhere dense class of finite graphs is stable, and conversely,
any stable class of finite graphs which is closed under taking 
subgraphs is nowhere dense. This result again manifests the notion of
nowhere denseness as a model-theoretic property. It also suggests 
stability theory as the framework for extending the border of 
tractability for first-order model-checking to dense graphs. 
This connection to stability theory has already been exploited for
the design of efficient algorithms on nowhere dense graphs. 
As a first step, it was used by Kreutzer et al.\ (SODA 2016) 
to strongly improve 
the parameter dependence between the equivalent notions of 
uniform quasi-wideness and nowhere denseness. Based on these
tools, Eickmeyer et al.\ (ICALP 2017) presented a near optimal 
kernelization algorithm for the distance-$r$ dominating set problem
on nowhere dense graph classes. We believe that this synergy 
between classical model theory and algorithms will lead to 
strong and very general algorithmic results. Our goal is to bring
together researchers from model theory and algorithms to 
explore these new connections between classical model theory
and algorithms. 
