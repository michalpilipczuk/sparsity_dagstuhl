
\heading{Personal Information}

\begin{small}
\noindent
\begin{tabular}{@{\hspace{0cm}}l @{\hspace{10mm}} p{13cm}}
{\bf Name and surname} & Michał Pilipczuk\\[0.1cm]
{\bf Date and place of birth} & June 25$^{\textrm{th}}$, $1988$, Warsaw, Poland\\[0.1cm]
%{\bf Citizenship} & Polish\\[0.1cm]
{\bf Address} & Institute of Informatics,\\
& Faculty of Mathematics, Informatics, and Mechanics of the University of Warsaw,\\
& ul. Banacha 2, 02-097 Warsaw, Poland\\[0.1cm]
%{\bf Phone number} & +48 22 5544458\\[0.1cm]
{\bf E-mail} & \verb+michal.pilipczuk@mimuw.edu.pl+\\[0.1cm]
{\bf Website} & \verb+http://www.mimuw.edu.pl/~mp248287+\\[0.1cm]
{\bf Research interests} & parameterized complexity, moderately exponential-time algorithms, kernelization,\\ &
logic in computer science, structural graph theory
\end{tabular}
\end{small}

\heading{Education and academic career}
\begin{small}
\noindent
\begin{tabular}{@{}p{3cm} @{\hspace{2mm}} p{13.2cm}}
since October 2015 & Assistant professor ({\em{pol.}} adiunkt) at the Institute of Informatics at the Faculty of Mathematics, Informatics, and Mechanics of the University of Warsaw, Poland. \\[0.2cm]
2014 --- 2015 & Postdoc of Warsaw Centre of Mathematics and Computer Science, affiliated with the Institute of Informatics, Faculty of Mathematics, Informatics, and Mechanics of the University of Warsaw,~Poland. \\[0.2cm]
2011 --- 2014 & {\em{Stipendiat}} (doctoral research fellow, a PhD position) at the Institute of Informatics of the University of Bergen, Norway, working under the supervision of prof. Fedor Fomin.
PhD thesis titled \emph{Tournaments and Optimality: New Results in Parameterized Complexity} defended in November 2013.\\[0.2cm]
%2011 --- 2013 & PhD Studies in Theoretical Computer Science under supervision of prof. Fedor V. Fomin, University of Bergen, Norway. \\[0.2cm]
2006 --- 2013 & Double Degree Program in Computer Science and Mathematics, Faculty of Mathematics, Informatics and Mechanics, University of Warsaw, Poland. 
Master thesis in computer science defended with honours in June 2011, master thesis in mathematics defended in April 2013.
%2003 --- 2006 & XIV Secondary School in Warsaw, mathematical-experimental program
\end{tabular}
\end{small}

\heading{Leadership and participation in grants}
\begin{small}
\noindent
\begin{tabular}{@{}p{3cm} @{\hspace{2mm}} p{13.2cm}}
since October 2017 & Researcher in ERC grant TOTAL ``Technology transfer between modern algorithmic paradigms'', led by Marek Cygan.\\[0.2cm]
2016 --- 2018 & Research partner in POLONEZ grant ``Algorithmic Structure Theory for Sparse Graphs'', led by Sebastian Siebertz
and funded by the National Science Center of Poland from the Horizon 2020 programme funds, Marie Sk\l{}odowska-Curie actions.\\[0.2cm]
2014 --- 2017 & Principal investigator of SONATA grant ``Optimality in Parameterized Complexity'', funded by the National Science Center of Poland.
\end{tabular}
\end{small}


\heading{Selected scientific awards}
\begin{small}
\noindent
\begin{tabular}{@{}p{1.8cm} @{\hspace{2mm}} p{14.2cm}}
2016 & ERCIM Cor Baayen Award 2016. Prize awarded annually to one promising young researcher in computer science and applied mathematics.\\[0.1cm]
2015, 2016 & START stipend granted by the Foundation for Polish Science (FNP), awarded with a~distinguishment for the highest ranked applications.\\[0.1cm]
2015 & Witold Lipski Prize for the best young researchers working in computer science in Poland.\\[0.1cm]
2015 & Stipend of the Ministry of Science and Higher Education of the Republic of Poland for outstanding young~researchers.\\[0.1cm]
2014 & Meltzer Prize for Young Researchers (Meltzerprisen for yngre forskere), awarded for the achievements in 2013.
\end{tabular}
\end{small}

\pagebreak

\heading{Selected invited talks}
\begin{small}
\noindent
\begin{tabular}{@{}p{1.5cm} @{\hspace{2mm}} p{14.7cm}}
Jun 2018 & Invited talk {\em{Parameterized algorithms for planar packing and covering problems using Voronoi diagrams}} at the workshop {\em{Fine-grained Complexity of Hard Geometric Problems}},
a satellite event of SoCG 2018. Budapest, Hungary.\\[0.1cm]
May 2018 & Invited talk {\em{From approximate to parameterized and back again: algorithms for geometric packing and covering problems using Voronoi diagrams}}
at the Lorentz Center workshop {\em{Fixed-Parameter Computational Geometry}}. Leiden, Netherlands.\\[0.1cm]
Aug 2017 & Invited talk {\em{On definable and recognizable properties of graphs of bounded treewidth}} at the 42$^{\text{nd}}$ International Symposium on Mathematical Foundations of Computer Science (MFCS 2017). Aarhus, Denmark.\\[0.1cm]
Jun 2015 & Invited talk {\em{Kernelization algorithms on sparse graph classes}} at the $5^{\textrm{th}}$ Workshop on Kernels. Nordfjordeid, Norway.\\[0.1cm]
Dec 2014 & Invited talks {\em{Algorithmic Lower Bounds based on ETH and SETH}} and {\em{Graph Isomorphism is FPT Parameterized by Treewidth}} at workshop {\em{Exact Algorithms and Lower Bounds}}, a satellite event of FSTTCS~2014. Delhi, India.\\[0.1cm]
\end{tabular}
\end{small}

\heading{Selected organization of scientific events}
\begin{small}
\noindent
\begin{tabular}{@{}p{1.6cm} @{\hspace{2mm}} p{14.6cm}}
Dec 2017 & Lecturer at the School on Recent Advances in Parameterized Complexity. Tel Aviv, Israel.\\[0.1cm]
Sep 2017 & Lecturer at the Parameterized Complexity Summer School, a satellite event of ALGO 2017. Vienna, Austria.\\[0.1cm]
Jul 2017 & Co-organizer (with Sebastian Siebertz) of a satellite workshop of ICALP 2018 on algorithms and structure for sparse graphs. Warsaw, Poland.\\[0.1cm]
Jun 2016 & Co-organizer (with Marek Cygan and Marcin Pilipczuk) of sessions on parameterized complexity during SIAM Conference on Discrete Mathematics 2016. Atlanta,~USA.\\[0.1cm]
Aug 2014 & Organizer and lecturer at the International School on Parameterized Algorithms. Bedlewo, Poland.
\end{tabular}
\end{small}

\heading{Service}
\begin{small}
\noindent {\bf{Supervised PhD students}}: Marcin Wrochna (graduated in 2018)
\vskip 0.1cm
\noindent {\bf{Program Committee member of}}: IPEC 2018 (co-chair), ICALP 2018 (track A), STOC 2018, STACS 2018, IPEC 2017, ESA 2016 (track A), WG 2016, ICALP 2015, WALCOM 2015, FSTTCS 2014
\end{small}

\heading{Selected publications connected with the proposal's topic}

\setcounter{pcount}{1}

\begin{footnotesize}
\paperbr{wide-stab}{J. Gajarsk\'y, S. Kreutzer, J. Ne\v{s}et\v{r}il, P. Ossona de Mendez, M. Pilipczuk, S. Siebertz, Sz. Toru\'nczyk}
{First-order interpretations of bounded expansion classes}{Proceedings of the 45$^{\text{th}}$ International Colloquium on Automata, Languages, and Programming, ICALP 2018}

\paper{wide-stab}{M. Pilipczuk, S. Siebertz, Sz. Toruńczyk}
{On the number of types in sparse graphs}{Proceedings of the 23$^{\textrm{rd}}$ Annual ACM/IEEE Symposium on Logic in Computer Science, LICS 2018}

\paperbr{succ-inv}{J. van den Heuvel, S. Kreutzer, M. Pilipczuk, D. A. Quiroz, R. Rabinovich, S. Siebertz}
{Model-checking for successor-invariant first-order formulas on graph classes of bounded expansion}{Proceedings of the 22$^{\textrm{nd}}$ Annual ACM/IEEE Symposium on Logic in Computer Science, LICS 2017}

\paperbr{nei-comp}{K. Eickmeyer, A. Giannopoulou, S. Kreutzer, O. Kwon, M. Pilipczuk, R. Rabinovich, S. Siebertz}
{Neighborhood complexity and kernelization for nowhere dense classes of graphs}{Proceedings of the 44$^{\text{th}}$ International Colloquium on Automata, Languages, and Programming, ICALP 2017}

\paperbr{bounded-exp}{P. G. Drange, M. S. Dregi, F. V. Fomin, S. Kreutzer, D. Lokshtanov, M. Pilipczuk, M. Pilipczuk, F. Reidl, F. Sánchez Villaamil, S. Saurabh, S. Siebertz, S. Sikdar}
{Kernelization and Sparseness: the case of Dominating Set}{Proceedings of the 33$^{\textrm{rd}}$ International Symposium on Theoretical Aspects of Computer Science, STACS 2016}
\end{footnotesize}
