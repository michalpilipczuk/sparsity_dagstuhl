\paragraph*{Expected outcomes.}
The main goal of the proposed seminar is to create a platform for the exchange of tools, ideas, and questions between researchers working on different aspects of sparsity.
So far, such synergy led to major developments due to the multi-disciplinary character of the field, and we hope that the seminar will foster new cross-field collaborations and lead to new results.
Besides this, the studied concepts of sparsity are very young and the new techniques are still not widely known in the related fields. 
This particularly applies to the field of algorithm design, where the toolbox seems to be applicable within multiple paradigms, whose respective communities are not aware of the new developments.
It would be desired if the seminar contributed to the visibility of the theory of sparse graph classes within neighboring areas and inspired new results based on its techniques.
Finally, the recent attempts at deploying theoretical sparsity-based methods in practice show promise and we believe that the community should strongly support applications-driven research.
We hope that the seminar will encourage more practical works such as implementation, evaluation, and fine-tuning of theoretical methods, and the usage of sparsity-based subroutines in larger systems.
