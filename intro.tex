\paragraph*{Introduction.}
It was realized already in the early days of computer science that structures (networks, databases, etc.) that are {\em{sparse}} appear ubiquitously in applications.
The sparsity of input can be used in a variety of ways, e.g. to design efficient algorithms. This motivates a theoretical study of the abilities and limitations of sparsity-based methods.
However, a priori it is not clear how to even define sparsity formally.
Focusing on graphs with bounded maximum degree seems too restrictive, as real-life networks tend to have high-degree hubs while retaining a moderate overall number of edges.
On the other hand, only requiring a constant upper bound on the average degree is too relaxed, as it allows the existence of dense substructures.
Another way is to concentrate on classes with topological constraints, e.g. planar graphs or, more generally, proper minor-closed classes,
or to generalize the example of bounded-degree networks to classes with bounded {\em{degeneracy}} (equivalently, bounded {\em{arboricity}}).
While these ideas inspired multiple results, it can be argued that they did not lead to the development of a robust mathematical theory.

In the late 2000s, Ne\v{s}et\v{r}il and Ossona de Mendez~\cite{NesetrilM08,NesetrilM08a,NesetrilM08b} proposed a different approach:
to introduce new definitions of uniform, structural sparsity for classes of graphs that would generalize the currently considered settings,
and to develop a toolbox of sparsity-based methods for such sparse classes.
The central notions they introduced are classes of {\em{bounded expansion}} and {\em{nowhere dense}} classes.
In a nutshell, a class of graphs $\Cc$ has bounded expansion if by contracting constant-radius connected subgraphs into single vertices
in any graph from $\Cc$ one cannot obtain graphs with arbitrary high average degree; for nowhere denseness, we only insist that in this way one cannot obtain arbitrary large cliques.
Thus, bounded expansion and nowhere denseness can be thought of uniform sparsity that persists under local modifications, modeled by constant-radius contractions.
Every class of bounded expansion is nowhere dense, and every class that excludes a fixed topological minor --- in particular every proper minor-closed class and every class of bounded maximum degree ---
has bounded expansion. Thus, these concepts are indeed broader than the previously considered settings, while sparse networks appearing in applications indeed tend to come from
classes of bounded expansion~\cite{DemaineRRVSS14}.

It quickly turned out that the proposed notions can be used to build a mathematical theory of sparse graph classes that offers a wealth of tools, leading to new techniques and powerful results.
Indeed, the last~10 years have seen a great progress in the area, witnessed by many publications in top venues.
It is particularly remarkable that the concepts of bounded expansion and nowhere denseness can be connected
to fundamental ideas from multiple other fields of computer science, often in a surprising way, and thus there are several complementary viewpoints on the subject.
On one hand, foundations of the area are grounded in {\em{structural graph theory}}, which aims at describing structure in graphs through various decompositions and auxiliary parameters.
On the other hand, nowhere denseness seems to delimit the border of expressibility and algorithmic tractability of first-order logic, which provides a link to {\em{finite model theory}}.
Finally, there is a fruitful transfer of ideas to and from the field {\em{algorithm design}}: sparsity-based methods can be used to design new, efficient algorithms,
and classic techniques of designing algorithms on sparse inputs inspire new combinatorial results on sparse graphs.
It is the synergy of these three fields that makes the theory of sparse graphs so exciting.
In the next paragraphs we explain how the recent progress can be viewed through the eyes of researchers working in these areas.

The aim of the proposed seminar is to ``stir in the pot'' and invite researchers working with structurally sparse structures to share their experience, questions, and ideas.
An important secondary goal is to facilitate discussion on practical applications of (so far) theoretical methods that were recently developed.
%We believe that the recent rapid progress in the area calls for capitalizing on the momentum.
