\paragraph*{Eyes of a combinatorialist.}

Graphs embeddable in the plane have always been understood as a prototype of a class of sparse graphs
both from the structural and algorithmic points of view.
However, the sparsity is not linked to purely quantitive parameters such as the number of edges:
an edge-subdivision of every graph has the number of edges linear in the number of vertices
but still possess complex properties of the original graph.
A great amount of research has been concerning the investigation of questions on structural properties
implying key combinatorial and algorithmic results linked to the sparsity of plane graphs.
Many such questions has been answered by the graph minor project of Robertson and Seymour,
which is published in a sequence of 23 papers totalling over 750 pages.
They provided a structural description of graphs avoiding specific graphs as a minor;
a particular example of such class of graphs is the class of graphs embeddable in the plane.
Many concepts that are now better understood as a part of this project are of high importance in computer science:
for example, the tree-width parameter, which measures how close the structure of a graph is to that of a tree,
plays an essential role in computer science in relation to designing algorithms for sparse inputs.

Recent years have brought substantial extensions of the results stemming from the graph minor project and
a new fascinating view on sparse graphs through lenses of the concepts of graph classes with bounded expansion and
nowhere-dense graph classes introduced by Ne\v est\v ril and Ossona de Mendez~\cite{}.
Many natural classes of sparse graphs, in particular minor-closed classes and classes of graphs with bounded degrees,
have bounded expanses and so fall into the framework developed by Ne\v est\v ril and Ossona de Mendez.
This framework turned out to be very robust since the graph classes captured by these notions
can be characterized in equivalent ways using several different means: low-treedepth colorings,
uniform quasi-wideness, existence of small separations, generalized coloring numbers, etc.
It soon became obvious that the concepts developed in this framework play an important role
in relation to algorithm design, in particular to tractability of first-order model checking~\cite{}.
This line of research culminated with a general first order model checking result of Grohe et al.~\cite{},
which in turn provided a new alternative description of nowhere-dense graph classes.

While the purely quantitive sparsity measures (such as the edge density) are not linked to any global structural properties,
the sparsity properties captured by the graph classes discussed above.
For example, one of the main results of the graph minor project is that
graphs avoiding a minor admit tree-like decompositions into pieces almost embeddable in surfaces of bounded genus.
Grohe and Marx~\cite{} gave a similar description for graph classes closed under topological containment and
Dvo\v r\'ak extended these results to graph classes closed under immersions.
Motivated by problems from algorithm design,
there is an intensive research on improving the dependence on parameters involved in these structural results.
In particular,
the recent simplified proof of the structure theorem in the graph minor setting~\cite{KawarabThomasWollan}
led to surprisingly good bounds,
which is of fundamental importance in relation to efficient fixed parameter tractable algorithms based on graph minors.
